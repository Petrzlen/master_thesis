%Ohurit komisiu
%Urobili vela prace, Nebolo to lahke

%TODO byt hrdy na svoju pracu pocas prezentacie 

%==OPONENT (konzultovat s veducim) 
%   snazime sa z inej katedry 
%   precita, svoj nazor, upozorni na chyby
%   (o com to je, hodnotiaca cast, zaverecne hodnotenie - otazky)
%   TODO! nemozem referencovat oponenta v prezentacii
%     - po prezentacii sa prejdu pripomienky oponenta,
%       idealne je mat to pripravene na slide-och 


% ==OBHAJOBA:  
%   ukazat, co sme sa naucili 
%   a ako to vieme aplikovat 
%   TODO! v prezentacii len to co sme urobil
   
%   mame cas dopracovat "mal som cas vysledky dorobit a opravit"
%     "nie je v praci" 
%     pomoze
%     dokument uz nemozem opravit

\documentclass[xcolor=dvipsnames]{beamer} 
\usepackage[slovak]{babel}
\usepackage[utf8]{inputenc}
\usepackage{hyperref}
\usepackage{tabularx} 

\usecolortheme[named=Plum]{structure} 
\usetheme[height=7mm]{Rochester} 
\setbeamertemplate{items}[ball] 
\setbeamertemplate{blocks}[rounded][shadow=true] 

\useoutertheme{umbcfootline} 

%%%%%%%%%%%%%%%%%%%%%%%%%%%%%%%%%%%%%%%%%%%%%%%%%%%%%%%%%%%%%%%%%%%%%%%%%
%%%%%%%%%%%%%%%%%%%%%%%%%%%%%%%%%%%%%%%%%%%%%%%%%%%%%%%%%%%%%%%%%%%%%%%%%
%Na úvodnej stránke uveďte: meno študenta, názov diplomovej práce a meno vedúceho diplomovej práce.
%Prezentácia by mala trvať asi 12 minút.
%Na stránky uvádzajte malý počet riadkov.
%Vyhýbajte sa používaniu žargónu.
%Používajte starú múdrosť: 1 obrázok je viac než 1000 slov.
%%%%%%%%%%%%%%%%%%%%%%%%%%%%%%%%%%%%%%%%%%%%%%%%%%%%%%%%%%%%%%%%%%%%%%%%%
%%%%%%%%%%%%%%%%%%%%%%%%%%%%%%%%%%%%%%%%%%%%%%%%%%%%%%%%%%%%%%%%%%%%%%%%%

% items enclosed in square brackets are optional; explanation below
\title[ANALYSIS OF A LEARNING ALGORITHM IN BIDIRECTIONAL NEURAL NETWORK]{
ANALYSIS OF THE GENERALIZED \\
RECIRCULATION-BASED LEARNING ALGORITHM \\
IN BIDIRECTIONAL NEURAL NETWORK \\
\vspace{3cm}
DIPLOMA THESIS
}
\author[P. Csiba]{Bc. Peter Csiba \\ Vedúci: doc. Ing. Igor Farkaš, PhD.}
\institute[FMFI UK]{
  UNIVERZITA KOMENSKÉHO V BRATISLAVE\\
  FAKULTA MATEMATIKY, FYZIKY A INFORMATIKY
}
\date{16-04-2014}

\begin{document}

%--- the titlepage frame -------------------------%
\begin{frame}[plain]
  \titlepage
\end{frame}


%%%%%%%%%%%%%%%%%%%%%%%%%%%%%%%%%%%%%%%%%%%%%%%%%%%%%%%%%%%%%%%%%%%%%%%%%
%%%%%%%%%%%%%%%%%%%%%%%%%%%%%%%%%%%%%%%%%%%%%%%%%%%%%%%%%%%%%%%%%%%%%%%%%
%V úvodnej časti prezentujte pojmy a kontext nevyhnutný pre formuláciu úloh riešených v diplomovej práci.
%Na stránky uvádzajte malý počet riadkov.
%Vyhýbajte sa používaniu žargónu.
%Používajte starú múdrosť: 1 obrázok je viac než 1000 slov.
%%%%%%%%%%%%%%%%%%%%%%%%%%%%%%%%%%%%%%%%%%%%%%%%%%%%%%%%%%%%%%%%%%%%%%%%%
%%%%%%%%%%%%%%%%%%%%%%%%%%%%%%%%%%%%%%%%%%%%%%%%%%%%%%%%%%%%%%%%%%%%%%%%%

\begin{frame}{Kontext práce - Neurónové siete - Perceptrón}
 
\begin{figure}[h]
  \centering
  \includegraphics[width=0.7\textwidth]{img/perceptron.pdf}    
  \caption{Perceptrón (McCulloch a Pitts, 1943). Označenie: $x$ je \emph{vstupný vektor} pre ktorý $x_0=1$; $w_{0k}$ je \emph{váhový vektor}; $\Sigma$ je \emph{sumačná funkcia}; $\eta_k$ je \emph{net}; $\phi$ je \emph{aktivačná funkcia}; $\theta_k$ je \emph{treshold}; $y_k$ je \emph{daný výstup} a $b_k$ je \emph{bias}.} 
  \label{fig:perceptron}
\end{figure} 

\end{frame}

\begin{frame}{Kontext práce - GeneRec}
  \begin{itemize} 
    \item GeneRec $:=$ Generalized Recirculation Algorithm {\tiny (O'Reilly, 1996)}.
  \end{itemize} 
  
  \begin{figure}
    \centering
    \includegraphics[scale=0.5]{img/3_layer_network_generec.png}
    %\caption{{\tiny A standard 3-layer network (Wikipedia).}} 
  \end{figure} 
  
  \vspace{-0.75cm}
  
  \begin{figure}
    \centering
    \scalebox{0.8}{
      \begin{tabular*}{1.2425\textwidth}{|cccc|}
        \hline
        Vrstva & Fáza & Net & Aktivácia\\
        \hline
        Vstupná ($s$)    & $-$ & - & $s_i = \mbox{vstup}$ \\
        \hline
        Skrytá ($h$)   & $-$ & \hspace{0.3cm}$\eta^{-}_j = \sum_i w_{ij}s_i + \sum_k w_{kj}o^{-}_k$\hspace{0.3cm} &
        $h^{-}_j = \sigma(\eta^{-}_j)$\hspace{0.3cm}\\
              &  +  & $\eta^{+}_j = \sum_{i}w_{ij}s_i + \sum_k w_{kj}o^{+}_k$ & $h^{+}_{j} = \sigma(\eta^{+}_j)$ \\
        \hline
        Výstupná ($o$) & $-$ & $\eta^{-}_k = \sum_j w_{jk}h_j$ & $o^{-}_k = \sigma(\eta^{-}_k)$\\
               &  +  & - & $o^{+}_k = \mbox{očakávaný výstup}$  \\
        \hline
      \end{tabular*}
    }
    %\caption{Aktivácie v modely GeneRec $:=$ Generalized Recirculation Algorithm (O'Reilly 1996).}
    \label{tab:generec}
  \end{figure}

\end{frame}

\newcommand{\Bx}{{\bf x}}
\newcommand{\By}{{\bf y}}
\newcommand{\Bh}{{\bf h}}
\newcommand{\Bw}{{\bf w}}
\newcommand{\Bc}{{\bf c}}

%Bidirectional Activation-based Learning algorithm (BAL) shares with GeneRec
%the phase-based activations and unit types, but differs from it by the connectivity
%that allows completely bidirectional associations to be established (GeneRec
%focuses on input-to-output mapping). Unlike GeneRec, BAL uses two pairs of
%weight matrices for each activation phase. In addition, in BAL we do not use
%dynamical settling process but compute the activations in one step as described
%in Table 2.
\begin{frame}{Kontext práce - BAL}
  \begin{itemize}
    \item BAL $:=$ Bidirectional Activation-based Neural Network Learning Algorithm (Farkaš a Rebrová, 2013).
    \item Učiace pravidlo: $ \frac{1}{\epsilon}\Delta w_{ij} = a_{i}^{-}(a_{j}^{+} - a_{j}^{-}). $
  \end{itemize}
  
  \vspace{-0.3cm}
  
  \begin{figure}[h!]  
    \centering
    \vspace{-5pt} 
    \includegraphics[scale=0.4]{img/3_layer_network_bal.png}
    %\caption{{\tiny A standard 3-layer network (Wikipedia).}} 
  \end{figure} 
  
  \vspace{-0.6cm}
  
  \begin{table}
    \centering
    \scalebox{0.78}{
      \begin{tabular*}{0.84\textwidth}{|cccl|}
        \hline
        Vrstva & Fáza & Net & Aktivácia\\
        \hline
        \Bx & F & - & $x^{\rm F}_i$ = stimulus\\ [1ex]
        \Bh & F & \hspace{0.3cm}$\eta^{\rm F}_j = \sum_i w^{IH}_{ij}x^{F}_i$\hspace{0.3cm} & $h^{\rm F}_j = \sigma(\eta^{\rm F}_j)$\hspace{0.3cm}\\ [1ex]
        \By & F & $\eta^{\rm F}_k = \sum_j w^{HO}_{jk}h^{F}_j$ & $y^{\rm F}_k = \sigma(\eta^{\rm F}_k)$\\ [1ex]
        \hline
        \By & B & - & $y^{\rm B}_k$ = stimulus\\ [1ex]
        \Bh & B & $\eta^{\rm B}_j = \sum_k w^{OH}_{kj}y^{\rm B}_k$ & $h^{\rm B}_j = \sigma(\eta^{\rm B}_j)$\\ [1ex]
        \Bx & B  & $\eta^{\rm B}_i = \sum_j w^{HI}_{ji}h^{\rm B}_j$ & $x^{\rm B}_i = \sigma(\eta^{\rm B}_i)$\\
        \hline
      \end{tabular*}
    }
    %\caption{Activation phases and states in the BAL model (Farkaš and Rebrová, 2013).}
    \label{tab:bal-states}
  \end{table}
\end{frame}


%%%%%%%%%%%%%%%%%%%%%%%%%%%%%%%%%%%%%%%%%%%%%%%%%%%%%%%%%%%%%%%%%%%%%%%%%
%%%%%%%%%%%%%%%%%%%%%%%%%%%%%%%%%%%%%%%%%%%%%%%%%%%%%%%%%%%%%%%%%%%%%%%%%
%V nadväznosti na úvodnú časť formulujte cieľ diplomovej práce.
%Na stránky uvádzajte malý počet riadkov.
%Vyhýbajte sa používaniu žargónu.
%Používajte starú múdrosť: 1 obrázok je viac než 1000 slov.
%%%%%%%%%%%%%%%%%%%%%%%%%%%%%%%%%%%%%%%%%%%%%%%%%%%%%%%%%%%%%%%%%%%%%%%%%
%%%%%%%%%%%%%%%%%%%%%%%%%%%%%%%%%%%%%%%%%%%%%%%%%%%%%%%%%%%%%%%%%%%%%%%%%

\begin{frame}{Cieľ diplomovej práce} 
  \begin{enumerate} 
    %\item Implement the GeneRec learning algorithm and test its properties using selected data sets.
    %\item Consider suitable modifications of the algorithm aimed at improving the network performance.
    %\item Analyse the convergence of the BAL network aimed at finding parameters on which the convergence depends. 
    \item Implementovať a analyzovať model BAL. 
    \item Zvážiť vhodné modifikácie algoritmu zamerané na úspešnosť siete. 
    \item Implementovať algoritmus GeneRec a porovnať ho s BALom. 
  \end{enumerate} 
  
  \begin{figure}[h!]  
    \centering
    \includegraphics[scale=0.40]{img/bal_performance.png}
    \caption{\small 4-2-4 enkóder: výsledky pre 100 sietí, počet úspešných behov~(vľavo), priemerný počet trénovacích epoch potrebných na konvergenciu~(vpravo), obe ako funkcie~$\lambda$. (Farkaš a Rebrová, 2013)} 
  \end{figure}  
  
\end{frame} 

%%%%%%%%%%%%%%%%%%%%%%%%%%%%%%%%%%%%%%%%%%%%%%%%%%%%%%%%%%%%%%%%%%%%%%%%%
%%%%%%%%%%%%%%%%%%%%%%%%%%%%%%%%%%%%%%%%%%%%%%%%%%%%%%%%%%%%%%%%%%%%%%%%%
%V ďalšej časti prezentujte vlastný prínos a vlastné výsledky porovnajte s výsledkami iných. Charakterizujte použité metódy.
%Na stránky uvádzajte malý počet riadkov.
%Vyhýbajte sa používaniu žargónu.
%Používajte starú múdrosť: 1 obrázok je viac než 1000 slov.
%%%%%%%%%%%%%%%%%%%%%%%%%%%%%%%%%%%%%%%%%%%%%%%%%%%%%%%%%%%%%%%%%%%%%%%%%
%%%%%%%%%%%%%%%%%%%%%%%%%%%%%%%%%%%%%%%%%%%%%%%%%%%%%%%%%%%%%%%%%%%%%%%%%

\begin{frame}{Vlastný prínos a vlastné výsledky}
  \begin{itemize}
    \item Navrhnutie a analýza modelu s rôznymi rýchlosťami učenia pre rôzne váhové matice.
    \item Zvýšenie pôvodnej úspešnosti BALu z 61\% na 94\%. 
  \end{itemize} 
  
  \begin{figure}[h!]  
    \centering
    \vspace{-8pt} 
    \includegraphics[scale=0.8]{img/success_to_lambdas.pdf}
  \end{figure} 
\end{frame}

\begin{frame}{Vlastný prínos a vlastné výsledky}
  \begin{itemize}
    \item Zrekonštruovanie výsledkov GeneRec a BAL. 
    \item Analýza atribútov siete BAL v priebehu učenia. 
    \item Zvýšenie úspešnosti pri výbere sietí so vzdialenejšími reprezentáciami na skrytej vrstve. 
  \end{itemize} 
  
  \begin{figure}[h!]  
    \centering
    \vspace{-8pt} 
    \includegraphics[scale=0.20]{img/compare_normal_and_hdist2.png}
    \caption{{\tiny $\lambda = \epsilon$ = rýchlosť učenia; $\sigma$ = disperzia normálnej distribúcie pôvodných váh}} 
  \end{figure} 
\end{frame}


%TODO legenda ku grafom 

\begin{frame}{Porovnanie s výsledkami iných}

  \begin{figure}[h!]  
    \centering
    \includegraphics[scale=0.45]{img/comparison_both.png} 
%    \begin{tabular}{cc}
%        \includegraphics[scale=0.4]{img/comparison.png} 
%      & 
%        {\tiny
%        \begin{tabular}{|l|l|l|l|}
%          \hline
%          Algorithm&$\epsilon = \lambda$&N=10000&Epcs\\
%          \hline
%          BAL orig &0.9&65/100&2000\\
%         \hline
%          BAL&0.7&5885&10000\\
%          \hline
%         BAL candidate&0.7&7024&30000\\
%          \hline
%          BAL long run&0.7&7934&500000\\
%          \hline
%        \end{tabular}
%        }
%      \\
%    \end{tabular} 
    \caption{Výsledky pre enkóder 4-2-4 kde $\epsilon$ je optimálna rýchlosť učenia, $N$ je počet sietí ktoré sa naučili daný problém, $Epcs$ je priemerný počet epoch potrebných na konvergenciu a $SEM$ je štandardná odchýlka. (O'Reilly,~1996)} 
    %{\tiny Results for the 4-2-4 encoder problem. $\epsilon$ is the optimal learning rate, $N$ is the number of networks that successfully solved the problem (out of 50), $Epcs$ is the mean number of epochs required to reach criterion, and $SEM$ is the standard error of this mean .}} 
  \end{figure} 
  \begin{center}
    
  \end{center} 
\end{frame}

\begin{frame}{Použité metódy - čo nefungovalo}
  \begin{itemize}
    \item Doteraz sme vyskúšali viacero variantov BALu: 
    \begin{itemize} 
      \item Iné učiace pravidlá, napr. pravidlo CHL {\tiny (Movellan, 1990)}.
      \item Momentum učenia. Dávkové učenie. Permutácia vstupov. 
      \item Vynechanie niektorých skrytých neurónov {\tiny (Dropout, Hinton, 2012)}. 
      \item Pridanie recirkulácie. Pridanie šumu.
      \item Dynamická rýchlosť učenia {\tiny (Yu a Chen, 1997)}. 
      \item Symetrická verzia. 
      \item Iné.
      \item Kombinácie predošlých. 
%      \item Cielené generovanie počiatočných váh. 
    \end{itemize} 
    \item V prípade 4-2-4 enkódera-a sme dosiahli 90-95\% úspešnosť, pričom Backpropagation má 100\% úspešnosť (jednosmerne). 
  \end{itemize} 
\end{frame} 

\begin{frame}{Analýza BAL - Zobrazenie priebehu}
  \begin{figure}[h!]  
    \centering
    \includegraphics[scale=0.4]{img/nice.png}
    \caption{{\small Zobrazenie priebehu skrytých aktivácií. Táto sieť bola {\bf úspešná}.}} 
  \end{figure} 
\end{frame} 

\begin{frame}{Analýza BAL - Zobrazenie priebehu}
  \begin{figure}[h!]  
    \centering
    \includegraphics[scale=0.4]{img/left_top.png}
    \caption{{\small Zobrazenie priebehu skrytých aktivácií. Táto sieť bola {\bf úspešná}.} }
  \end{figure} 
\end{frame}

\begin{frame}{Analýza BAL - Zobrazenie priebehu}
  \begin{figure}[h!]  
    \centering
    \includegraphics[scale=0.4]{img/tazisko.png}
    \caption{{\small Zobrazenie priebehu skrytých aktivácií. Táto sieť bola {\bf neúspešná}.} }
  \end{figure} 
\end{frame}

\begin{frame}{Analýza BAL - Zobrazenie priebehu}
  \begin{figure}[h!]  
    \centering
    \includegraphics[scale=0.4]{img/non-convergent.png}
    \caption{{\small Zobrazenie priebehu skrytých aktivácií. Táto sieť bola {\bf neúspešná}.}} 
  \end{figure} 
\end{frame}


%%%%%%%%%%%%%%%%%%%%%%%%%%%%%%%%%%%%%%%%%%%%%%%%%%%%%%%%%%%%%%%%%%%%%%%%%
%%%%%%%%%%%%%%%%%%%%%%%%%%%%%%%%%%%%%%%%%%%%%%%%%%%%%%%%%%%%%%%%%%%%%%%%%
%Na záver sformulujte možnosti ďalšieho rozpracovania práce.
%Na stránky uvádzajte malý počet riadkov.
%Vyhýbajte sa používaniu žargónu.
%Používajte starú múdrosť: 1 obrázok je viac než 1000 slov.
%%%%%%%%%%%%%%%%%%%%%%%%%%%%%%%%%%%%%%%%%%%%%%%%%%%%%%%%%%%%%%%%%%%%%%%%%
%%%%%%%%%%%%%%%%%%%%%%%%%%%%%%%%%%%%%%%%%%%%%%%%%%%%%%%%%%%%%%%%%%%%%%%%%

\begin{frame}{Možnosti ďalšieho rozpracovania práce}
  \begin{itemize} 
    \item Inicializácia váh s určitými vlastnosťami.
    \item Matematická formulácia aproximácie dynamického systému BAL a skúmanie jeho konvergencie. 
    \item Urýchlenie konvergencie pomocou momentumu. 
  \end{itemize} 
\end{frame} 

\begin{frame}{Priestor na otázky}
  \begin{center}
  \includegraphics[scale=0.75]{img/question.png}
  \end{center}
\end{frame}

\begin{frame}{Ďakujem za pozornosť!}
  \begin{center}
{\bf Ďakujem za pozornosť!} 
  \end{center}
  
  \vspace{2cm}
  
  \begin{center}
  Aktuálna verzia: \url{https://github.com/Petrzlen/master_thesis} \\
  \vspace{5mm} 
  \small{P.S. Ospravedlňujeme sa za kvalitu obrázkov (TODO pdf).}
  \end{center}
\end{frame}

\end{document}

