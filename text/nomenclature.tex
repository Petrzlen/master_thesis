\section*{Dictionary}
\markboth{DICTIONARY}{}    
\addcontentsline{toc}{section}{Dictionary}

\begin{itemize}
\item Differential equations - TODO learn the basics (to have an intuition for computing the learning rules). Continuous? Ask Ondrac for materials. 
\item Difference equations - TODO learn the basics (to have an intuition for computing the learning rules). Discrete? 
\item Antiparallel vectors - (Wiki, TODO) In a vector space over $\mathbb{R}$ (or some other ordered field), two nonzero vectors are called antiparallel if they are parallel but have opposite directions. In that case, one is a negative scalar times the other.
\item Kronecker delta - (Wiki, TODO) In mathematics, the Kronecker delta or Kronecker's delta, named after Leopold Kronecker, is a function of two variables, usually integers. The function is 1 if the variables are equal, and 0 otherwise: 
$$
    \delta_{ij} = \left\{\begin{matrix} 0, & \mbox{if } i \ne j \\ 1, & \mbox{if } i=j, \end{matrix}\right. $$

\item Steady state = fixed point. (Wiki, TODO) An important goal is to describe the fixed points, or steady states of a given dynamical system; these are values of the variable which won't change over time. Some of these fixed points are attractive, meaning that if the system starts out in a nearby state, it will converge towards the fixed point.

\item Periodical points. (Wiki, TODO) Similarly, one is interested in periodic points, states of the system which repeat themselves after several timesteps. Periodic points can also be attractive. Sharkovskii's theorem is an interesting statement about the number of periodic points of a one-dimensional discrete dynamical system.

\item Final mean root square weight per connection (Pineda, page 3). 

\item PCA - Principial component analysis. TODO. \cite{hinton1988learning} 

\item Auto encoder. TODO. \cite{hinton1988learning} 

\item Mean field annealing. Mean field annealing ( Soukoulis et al., 1983; Bilbro et al., 1989) is a deterministic approximation to simulated annealing which is significantly more computationally efficient (faster) than simulated annealing ( Bilbro et al., 1992). Instead of directly simulating the stochastic transitions in simulated annealing, the mean (or average) behavior of these transitions is used to characterize a given stochastic system. Because computations using the mean transitions attain equilibrium faster than those using the corresponding stochastic transitions, mean field annealing relaxes to a solution at each temperature much faster than does stochastic simulated annealing. \url{http://neuron.eng.wayne.edu/tarek/MITbook/chap8/8_4.html}

\end{itemize}

