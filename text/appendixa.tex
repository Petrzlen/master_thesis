\section*{Appendix A}
\appendix
\addcontentsline{toc}{section}{Appendix A}
\markboth{Appendix A}{}

\subsection{Dynamical systems theory}
TODO: Cite the basic theory (Jerome R. Busemeyer (2008), "Dynamic Systems". To Appear in: Encyclopedia of cognitive science, Macmillan. Retrieved 8 May 2008)

(Wiki, TODO) Dynamical systems theory is an area of mathematics used to describe the behavior of complex dynamical systems, usually by employing differential equations or difference equations. When differential equations are employed, the theory is called continuous dynamical systems. When difference equations are employed, the theory is called discrete dynamical systems. When the time variable runs over a set which is discrete over some intervals and continuous over other intervals or is any arbitrary time-set such as a cantor set then one gets dynamic equations on time scales. Some situations may also be modelled by mixed operators such as differential-difference equations.

This theory deals with the long-term qualitative behavior of dynamical systems, and the studies of the solutions to the equations of motion of systems that are primarily mechanical in nature; although this includes both planetary orbits as well as the behaviour of electronic circuits and the solutions to partial differential equations that arise in biology. Much of modern research is focused on the study of chaotic systems.
