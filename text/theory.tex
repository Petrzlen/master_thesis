% The theory and concepts of your work. For example, if you work on compiler you can mention about how compiler works without getting much into detail.

\newcommand{\argmin}{\operatornamewithlimits{arg\,min}}
\newcommand{\Bx}{{\bf x}}
\newcommand{\By}{{\bf y}}
\newcommand{\Bh}{{\bf h}}
\newcommand{\Bw}{{\bf w}}
\newcommand{\Bc}{{\bf c}}

In this section we will describe the basics of artificial neural networks. We will also introduce the notation used in this work. Note that the definitions and notations vary through the literature. We use the one which the author is familiar with. For the reader who is comfortable with this topic we recommend to skip this section and go to related models \ref{overview-models}. 

%=============================================================
\subsubsection{Perceptron.}

The theory behind artificial neural networks started with the model of \emph{Perceptron} introduced by \citet{mcculloch1943logical}. It is a simple model which transforms a vector of inputs $s$ to an output value $y$. 

\begin{figure}[h]
  \centering
  \includegraphics[width=0.7\textwidth]{img/perceptron.pdf}    
  \caption{Perceptron. Notation: $x$ is the \emph{input vector} where always $x_0=1$, $w_{0k}$ is the \emph{weight} vector, $\Sigma$ is the \emph{summing} junction, $\eta_k$ is the \emph{net input}, $\phi$ is the \emph{activation function}, $\theta_k$ is the \emph{treshold}, $y_k$ is the \emph{output} and $b_k$ is the \emph{bias}.} 
  \label{fig:perceptron}
\end{figure}

The whole transformation of the input vector to the output activation could be written as follows: 
\begin{equation}
\label{eq:perceptron} 
y_k =
\left\{
	\begin{array}{ll}
		0 & \mbox{if } \phi(\sum_{i=0}^N x_iw_{ik}) < \theta_k \\
		1 & \mbox{otherwise}
	\end{array}
\right.
\end{equation} 

Equation~\ref{eq:perceptron} describes a simple \emph{binary treshold perceptron}. One could observe that the binary perceptron divides the vector space $\mathbb{R}^N$ by a $(n-1)$--dimensional hyperplane. This behaviour was studied by \citet{rosenblatt1958perceptron}. Now we see the importance of bias which is the absolute term in the equation of the hyperplane. 

\paragraph{Continuous perceptron.}
We put additional constraints for the activation function $\phi \mathbb{R} \mapsto (0,1)$: 
\begin{enumerate} 
\item It is differentiable and monotonously increasing,
\item and satisfying two asymptotic conditions $t(-\infty)=0$ and $t(\infty)=1$. 
\end{enumerate} 
Usually, the transfer function is realized by the logistic function: 

\begin{equation}
\frac{1}{1 + e^{-\eta}}. 
\end{equation} 

To allow the outputs to be from range $(0,1)$ we drop the treshold and simply output $\phi(\eta_k)$. 

\paragraph{Learning.} 
The goal of a perceptron is to \emph{learn} a given set $T = \{(X^j, t_j)\}$ mappings, where $X^j$ is the input vector $(x_{j0},x_{j1}, \ldots, x_jN)$ and $t_j$ is the corresponding target. Ideally, we want from the perceptron to be able generalize for novel inputs. The goal is formalized by minimizing an error function: 

\begin{equation}
\label{eq:perceptron-error} 
E = \sum_{k=1}^{N} \frac{1}{2}(y_k-t_k)^2.
\end{equation} 

A straightforward method to achieve this is simply updateing weights according to the partial derivates of the error function: 

\begin{equation}
\label{eq:perceptron-learning} 
\frac{\partial E}{\partial w_{ik}} = (y_k - t_k)\phi'(\eta_k)x_i = (y_k - t_k)y_k(1 - y_k)x_i.
\end{equation} 

From equation~\ref{eq:perceptron-learning} we can develop a following learning algorithm: 
\begin{figure}[h]
  \centering
\begin{lstlisting}
for e from 1 to EPOCHS: 
  foreach (X^j, t_j) in T: 
    calculate y_j
    for i from 0 to N: 
      update w_{ij}
\end{lstlisting}
  \caption{Perceptron learning. Developed from equations~\ref{eq:perceptron-error} and \ref{eq:perceptron-learning} we get an \emph{weight update} rule which is applied in loop for element in $T$. TODO sth better than lstlisting} 
  \label{fig:perceptron-learning}
\end{figure}

Usually, the \emph{update rule} is written as: 
\begin{equation} 
\Delta w_{ik} = \lambda (y_k - t_k)y_k(1 - y_k)x_i,
\end{equation} 
where $\lambda$ is the \emph{learning speed}. 

 
\label{sec:perceptron} 

%=============================================================
\subsubsection{Multilayer Feedworward Networks} 
\label{sec:multilayer} 

We will define multilayer feedforward networks as in \citet{haykin1994neural}. First, we define a \emph{layered} neural network where neurons are organised to form layers. In the simplest version we have an \emph{input layer} of source nodes and an \emph{output layer} which is formed by aforementioned perceptrons. In other words this is a \emph{feedforward} or \emph{acyclic} type of network as the \emph{activation}, i.e. outputs of the neurons are computed from the input to the output layer and never \emph{backwards}. 

Multilayer neural network has one or more \emph{hidden layers} in addition to the input and ouput layer as shown on figure~\ref{fig:multilayer}: 

\begin{figure}[H]
  \centering
  \includegraphics[width=0.6\textwidth]{img/multilayer.pdf}    
  \caption{Fully connected feedforward multilayer with one hidden layer and one output layer. The source nodes supply the activation pattern, i.e. input vector, which is applied to second layer neurons. The output signal of the hidden layer is used as the input for the output layer. As shown by \citet{cybenko1989approximation} the three layer network is an universal approximator continuous functions on compact subsets of $\mathbb{R}^n$.} 
  \label{fig:multilayer}
\end{figure}

There exists several methods for training multilayer networks. First, we will describe the most common Backpropagation in \ref{models-bp} and then methods related to our work such as CHL \ref{models-chl}, GeneRec \ref{models-generec} and BAL \ref{models-bal}. 

\subsubsection{Recurrent Networks}

Recurrent networks arise problems with computing their activations. For example imagine a cycle of neurons. That means that output of a particular unit could affect its input. Therefore the activations in general couldn't be computed only by one forward pass. This introduces real--valued dynamic systems for computing the activations. We can observe that it holds that $\frac{\partial\eta}{\partial t} = 0$ for the activations of neurons in the fixed point state. There are several approaches solving these dynamic systems and deriving the learning rule \cite{pineda1987generalization, pearlmutter1989learning, williams1989learning, elman1990finding, haykin1994neural}. 

\begin{figure}[H]
  \centering
  \includegraphics[width=0.6\textwidth]{img/theory-recurrent.png}    
  \caption{Simple Recurrent network designed by \citet{elman1990finding}. Image from \citep{haykin1994neural}.} 
  \label{fig:theory-recurrent}
\end{figure}

An \emph{iterative method} is used by \citet{movellan1990contrastive} for computing activations. In the first step the input neurons have activations equal to the input vector and the other neurons have zero activation. In the next steps activations from the last step are used to compute activation in this step as shown in equation~\ref{eq:chl-activation}. This rule is iterated while the activation are settled. For particullar symmetric network it could be proved that activations will converge \citep{o1996bio}. For more general networks a dynamic system based on rule~\ref{eq:chl-activation} could be derived and a fixed point solution could be found by solving a set of non--linear equations (TODO ref). \citet{movellan1990contrastive} proposes using the method of simulated annealing \citep{kirkpatrick1983optimization,vcerny1985thermodynamical} to improve the learning rule and to avoid settling the network in a local minima. We experimented with the iterative method for a two--way GeneRec \ref{sec:our-bal-recirc}. 

\begin{equation}
\label{eq:chl-activation} 
\eta_i(t+1) = \phi(\sum_j w_{ji}\eta_i(t)
\end{equation}
