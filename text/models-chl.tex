\subsubsection{Contrastive Hebbian Learning}
\label{models-chl} 

%TODO write it as a comparison to BAL and GeneRec, i.e. put it into context 

The main idea of \emph{Contrastive Hebbian Learning} developed by \citet{movellan1990contrastive} is to have two activation phases in an aribtrary Hopfield network \citep{hopfield1984neurons}. In the first phase, called \emph{minus phase} and denoted \quotes{-}, only the input vector is \emph{clamped}, i.e. presented to the underlying network. In the second phase, called \emph{plus phase} and denoted \quotes{+}, both the input and target are clambed to the underlying network. The learning is based on the difference of these two activations. 

%TODO: Check if correct 
\begin{table}
  \centering
  \begin{tabular}{|cccc|}
    \hline
    Layer & Phase & Net Input & Activation\\
    \hline
    Input (s)    & $-$ & - & $s_i$ = stimulus input\\
    \hline
    Hidden (h)   & $-$ & \hspace{0.3cm}$\eta^{-}_j = \sum_i w_{ij}s_i + \sum_k w_{kj}o^{-}_k$\hspace{0.3cm} &
    $h^{-}_j = \sigma(\eta^{-}_j)$\hspace{0.3cm}\\
          &  +  & $\eta^{+}_j = \sum_{i}w_{ij}s_i + \sum_k w_{kj}o^{+}_k$ & $h^{+}_{j} = \sigma(\eta^{+}_j)$ \\
    \hline
    Output (o) & $-$ & $\eta^{-}_k = \sum_j w_{jk}h_j$ & $o^{-}_k = \sigma(\eta^{-}_k)$\\
           &  +  & - & $o^{+}_k$ = target output \\
    \hline
  \end{tabular}
  \caption{Equilibrium network variables in CHL model \citep{movellan1990contrastive}.}
  \label{tab:chl-states}
\end{table}


Based on Hopfield energy function defined on figure~\ref{fig:models-chl-hopfield} a learning rule is derived by \citet{movellan1990contrastive}: 
\begin{equation}
\label{eq:models-chl-learning-rule}
\Delta w_{ij} = a_i^{+}a_j^{+} - a_i^{-}a_j^{-}
\end{equation}
It could be shown that the learning rule \ref{eq:models-chl-learning-rule} decreases the Hopfield energy function \ref{eq:models-chl-hopfield-E} \citep{movellan1990contrastive}. Moreover it could be shown that the CHL learning rule is equivalent to Backpropagation while it is biologically more plausible \citep{o1996bio, o2001generalization}. 

\begin{figure}[H]
  \centering
  \begin{equation}
    \label{eq:models-chl-hopfield-E}
    E = -\frac{1}{2}\sum_i\sum_ja_iw_{ij}a_j
  \end{equation} 
  \begin{equation}
    \label{eq:models-chl-hopfield-S}
    S = \sum_i \int_{rest}^{a_i} f_i^{-1}(a)da
  \end{equation} 
  \caption{Definition of a continuous Hopfield Energy function $F = E + S$, where $a^T = [a_1,\ldots,a_n]$ is the activation vector, $f_i$ is bounded, monotically increasing, differentiable activation function \citep{movellan1990contrastive}.} 
  \label{fig:models-chl-hopfield}
\end{figure}
   
