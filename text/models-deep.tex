\subsubsection{Deep architectures}
Theoretical results strongly suggest that in order to learn the kind of complicated functions that can represent high-level abstractions (e.g. in vision, language, and other AI-level tasks), one needs deep architectures. Deep architectures are composed of multiple levels of non-linear operations, such as in neural nets with many hidden layers or in complicated propositional formulae re-using many sub-formulae. Searching the parameter space of deep architectures is a difficult optimization task, but learning algorithms such as those for Deep Belief Networks have recently been proposed to tackle this problem with notable success, beating the state-of-the-art in certain areas. This paper discusses the motivations and principles regarding learning algorithms for deep architectures, in particular those exploiting as building blocks unsupervised learning of single-layer models such as Restricted Boltzmann Machines, used to construct deeper models such as Deep Belief Networks \cite{bengio2009learning} (Abstract copied).

TODO read the article. 
