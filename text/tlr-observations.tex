
\subsection{Observations}
\label{sec:results-two-lambdas}  %TODO (still relevant label? ) 

Other partial results we found interesting. 
TODO talk about these points (item -> paragraph ; item -> other results): 

\subsubsection{Momentum}
\label{sec:results-momentum} 

\ref{sec:our-momentum} 
%======== (3D) L1 x L2 x patSuccF : TLR vs. best momentum =========
%======== (3D) L1 x L2 x epochs : TLR vs. best momentum =========
%======== (2D) best TLR on ALL_SUCC x epoch (std-dev) : TLR vs. best momentum ==========

\subsubsection{Weight initialization}
\label{sec:results-sigma} 

\ref{sec:our-sigma} 
%======== (3D) L1 x L2 x patSuccF : TLR vs. best sigma =========
%======== (3D) L1 x L2 x epochs : TLR vs. best sigma =========
%======== (2D) best TLR on ALL_SUCC x epoch (std-dev) : TLR vs. best sigma ==========

\subsubsection{Long run} 
\label{sec:results-long-run} 

(+ 10\%). It seems it's always better to have long runs. even after 800,000 epochs there are some networks for which the error change

%======== (3D) L1 x L2 x patSuccF : TLR vs. long run best =========
%======== (3D) L1 x L2 x epochs : TLR vs. long run best =========
%======== (2D) best TLR on ALL_SUCC x epoch (std-dev) : TLR vs. long run best ==========


\subsubsection{Other}

\begin{itemize} 
\item Matrices IH-OH and HO-HI tend to be same in autoassociative tasks. 
\item Hidden representation distance is a meaningful measure (LinReg on pre\_measure) (+ 10\%) . 
\item In triangle (non-convex). Not-in triangle is a must to condition for success. \\
TODO rerun to get "1 err in\_triangle" with and without preselection (there was a bug in the old data). 
\item reprezentacie na hidden absolutne rovnake (forward, backward), matice rozne:
\item Convergence Epsilon - weights tend to infinity (09-12-2013: Convergence which depends on average weight change does not work). 
\item Rerun - same config, different order when training 
All bad: \\
err sigma lambda momentum success sample\_ratio \\
0.0 2.3 0.7 0.0 19.296918767507005 6889/35700 \\
1.0 2.3 0.7 0.0 68.05602240896359 24296/35700 \\
2.0 2.3 0.7 0.0 12.644257703081232 4514/35700 \\
3.0 2.3 0.7 0.0 0.0028011204481792717 1/35700 \\

All good:  \\
err sigma lambda momentum success sample\_ratio \\
0.0 2.3 0.7 0.0 99.98911353032659 64293/64300 \\
1.0 2.3 0.7 0.0 0.01088646967340591 7/64300 \\
\end{itemize}
 
