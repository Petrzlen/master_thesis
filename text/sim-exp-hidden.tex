
%==========================================================
\subsubsection{Hidden activations}
\label{sec:our-hidden-activation} 

We observed that hidden activations in BAL \ref{sec:models-bal} tend to settle fast as shown on figure~\ref{fig:results-candidates-dist-h-fb}, i.e.~the weight changes become close to zero because $|H^F - H^B| \approx \overrightarrow{0}$. Therefore the network is \emph{de--facto} reduced to a two--layer network between the constant hidden activations and the target values. Thus \emph{at least} all cases when the hidden activations are not linearly separable it is impossible for $w_{HI}$ and $w_{HO}$ to learn targets. This behaviour is demonstrated by the \emph{in\_triangle} measure \ref{sec:our-in-triangle}. 

A more formal explaination why the hidden activations tend to settle fast could be given by the GeneRec learning rule \ref{eq:models-generec-learning-rule}: 
\begin{equation} 
  \Delta w_{ij} = a_i(b_j - a_j),
\end{equation} \nonumber 
which for the $W^{IH}$ and $W^{OH}$ yields: 
\begin{align} 
  \Delta w_{ij}^{IH} &= x^F_i(h^B_j - h^F_j) \nonumber \\ 
  \Delta w_{ij}^{OH} &= y^B_i(h^F_j - h^B_j). \nonumber  
\end{align} 
We see that both terms $(h^B_j - h^F_j)$ and $(h^F_j - h^B_j)$ push $W^{IH}$ and $W^{OH}$ to settle $h^B_j = h^F_j$. This experiment was one of the reasons we started experimenting with dynamic learning rate \ref{sec:our-dynamic-lambda} what lead to the TLR model \ref{sec:our-tlr}. 

