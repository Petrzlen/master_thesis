\subsubsection{Candidate selection} 
\label{sec:sim-exp-candidates}

The \emph{candidate selection} model was used to test and confirm if some particular network \emph{features}~(\ref{sec:our-candidates-features}) have an impact to the overall network performance. The only difference between standard BAL~(\ref{sec:models-bal}) and candidate selection is that before the training phase, $N$ networks are randomly generated from which a \emph{best candidate network} is selected. For selecting the best candidate we use \emph{feature function} defined as $F: \mbox{network} \mapsto \mathbb{R}$. 

\begin{algorithm}[H]
%  \KwData{$candidates$ }
  \begin{algorithmic}
    \State $best\_candidate \gets (\infty, null)$
    \For{$i = 1$ to $N$} 
      \State $gn \gets generate\_network()$
      \State $candidate \gets (F(gn), gn)$
      \If{$candidate < best\_candidate$}
        \State $best\_candidate \gets candidate$
      \EndIf
    \EndFor
  \caption{Candidate selection pseudocode.}
  \label{alg:our-candidates-pseudocode} 
  \end{algorithmic}
\end{algorithm} 

\paragraph{Features.}
\label{sec:our-candidates-features}

We denote $X_I$, $H_I$ and $Y_I$ as the front, hidden and back activation \emph{vector}s for input $I$ (and the corresponding target). For the rest of the notation please consult Section~\ref{tab:models-bal-activation}. We measured the following \emph{features}: 

\begin{itemize} 
\label{sec:our-h-dist} 
\item $dist_{H}$ (real) -- the average distance between all hidden activations of the inputs, i.e.~$avg_{I \neq J}\left(dist(H_I^{+},H_J^{+})\right)$. %forward / backward! 

\label{sec:our-dist-h-fb}
\item $dist_{H}^{FB}$ (real) -- the average distance between corresponding forward and backward \emph{hidden} activations, i.e.~$avg_{I}\left(dist(H_I^{-},H_I^{+})\right)$.

\label{sec:our-dist-o-fb}
\item	$dist_{V}^{FB}$ (real)-- the average distance between corresponding forward and backward \emph{visible} activations, i.e.~  $avg_{I}\left(dist(Y_I^{-},X_I^{+})\right)$. Note that this feature is only relevant for auto--associative tasks such as 4-2-4 encoder~(\ref{sec:datasets-auto4}). 

\label{sec:our-m-wei}
\item $matrix\_weight$ (real) -- average weight of the network, i.e. average value of all weight matrices $W^{IH}$, $W^{HO}$, $W^{OH}$ and $W^{HI}$. Note that each matrix value has the same impact to \emph{matrix\_weight}. 

\label{sec:our-in-triangle}
\item $in\_triangle$ (bool) -- check if hidden activations of inputs  $H_I^{+}$ form a convex polygon, i.e.~if the hidden activation points are all \emph{lineary separable}~(\ref{sec:models-perceptron}). Therefore $in\_triangle=0$ is a necessary condition for perfect success rate. Consult Figure~\ref{fig:results-hidden-activations-bal} for examples of convex and non--convex hidden activations. Note that $in\_triangle$ was implemented only for hidden size equal to two, i.e. for the 4-2-4 encoder task. 

\label{sec:our-fluctuation}
\item $fluctuation$ (real) -- the maximal difference between activations in the last two iterations when using the \emph{iterative method} for activation computation~(\ref{eq:models-generec-activation}). With other words, let $a_i(t)$ be the activation of unit $i$ in iteration $t$ and $T$ be number of iterations. Then $fluctuation$ is $max_i|a_i(T-1) - a_i(T)|$. So if $fluctuation \approx 0$ then the iterative method was successful and all activations settled.
\end{itemize} 

Also all \emph{parameters} of TLR were included such as $\lambda_v$, $\lambda_h$~(\ref{sec:our-tlr}), momentum $\mu$~(\ref{sec:our-momentum}) and weight distribution $\sigma$~(\ref{sec:our-sigma}). 

\paragraph{Linear regression.}
To get the most important features we trained a feature function on a \emph{feature dataset} consisting of individual features and with \emph{label} equal to $bitErr^F = 1-bitSucc^F$. The dataset was created by generating standard BAL networks, measuring feature values before the training phase and adding the success rate label after the training phase. On this feature dataset we trained a simple linear regression model shown in equation~(\ref{eq:results-candidates-linear-regression}).
\begin{align} 
\label{eq:results-candidates-linear-regression} 
bitErr^F &= 
- 0.328 \times dist_{H}
+ 0.140 \times dist_{H}^{FB}
- 0.100 \times dist_{V}^{FB} \nonumber \\
&+ 0.019 \times matrix\_sim
- 0.127 \times \sigma
+ 0.000 \times matrix\_weight
+ 3.610
\end{align}  
From equation~(\ref{eq:results-candidates-linear-regression}) we observe that the feature which contributed the most to $patSucc^F$ is $dist_{H}$. This was used as a inspiration for the TLR model~(\ref{sec:our-tlr}). Furthermore, we started using the feature function~(\ref{eq:our-fitness-function}) for all candidate selection simulations. It simply choses the network with greatest $dist_{H}$. For the 4-2-4 encoder task we also add $in\_triangle$ as primary feature to ensure the convexity of initial hidden activations. 
\begin{equation} 
\label{eq:our-fitness-function} 
  F(network) = -dist_{H}(network)
\end{equation} 

%\paragraph{Results.}
%\label{sec:results-candidates} 

%\ref{fig:results-tlr-auto4-epoch} 


