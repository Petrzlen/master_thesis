
\subsection{Complex Binary Vector Associations} 

\ref{sec:datasets-k3} 


\subsubsection{Two learning rate} 
\label{sec:tlr-k3}
%Main purpose of this dataset is to test different hidden sizes 
%======== (4D) L1 x L2 x patSuccF x hidden size =========
%======== (4D) L1 x L2 x epochs x hidden size =========

\begin{figure}[H]
  \centering
  \includegraphics[width=0.45\textwidth]{img/placeholder.png}   
  \includegraphics[width=0.45\textwidth]{img/placeholder.png}    
  \caption{TLR performance on the \emph{CBVA} task with different hidden sizes.}
  \label{fig:results-tlr-k3-success}
\end{figure}

%======== (3D) best TLR on ALL_SUCC x epoch (std-dev) x hidden size ==========
\begin{figure}[H]
  \centering
  \includegraphics[width=0.45\textwidth]{img/placeholder.png}    
  \caption{TLR success evolution for the \emph{CBVA} task.}
  \label{fig:results-tlr-k3-epoch} 
\end{figure}

\subsubsection{Comparison} 
\label{sec:results-cmp-k3} 

%===== TODO table: best parameter setting networks with hidden.size= constant (success, epoch, stddev) / model \\

\begin{table}
  \centering
    \begin{tabular}{|l|l|l|l|l|}
    \hline
    Algorithm&$\lambda$&Success&Epcs&SEM \\
    \hline
    \end{tabular}
  \caption{Comparing performance of different models on the \emph{CBVA} task.} 
  \label{tab:results-cmp-k3}
\end{table}


\subsubsection{Different hidden sizes} 
\label{sec:results-dhs-k3} 

TODO ref 

