%11.   State of the Art (or Related Work or Literature Review)
%In this section you should present the theoretical basis of your work and overview of the existing solutions. When discussing on the exiting solutions you should relate and qualitatively compare them with yours. This chapter should contain:

% 1) The theory and concepts of your work. For example, if you work on compiler you can mention about how compiler works without getting much into detail.

% 2) Existing state of the art solution. For example, if you implement an optimization pass, you should mention about existing optimization passes that are relevant with yours. You should emphasize the commons and differences with your solution and with the existing ones.

% You may divide this chapter in sections (e.g. each for the different existing solution).

\section{Overview}
\label{sec:overview} 

\subsection{Theory and Concepts}
\label{sec:theory} 

% The theory and concepts of your work. For example, if you work on compiler you can mention about how compiler works without getting much into detail.

\newcommand{\argmin}{\operatornamewithlimits{arg\,min}}
\newcommand{\Bx}{{\bf x} }
\newcommand{\By}{{\bf y} }
\newcommand{\Bh}{{\bf h} }
\newcommand{\Bw}{{\bf w} }
\newcommand{\Bc}{{\bf c} }

In this section,we will describe the basics of artificial neural networks. We will also introduce the notation used in this work. Note that the definitions and notations vary through the literature. We use the one which the author is familiar with. For the reader who is comfortable with this topic we recommend to skip this section and go to related models~(\ref{sec:overview-models}). 

%=============================================================
\subsubsection{Perceptron.}

The theory behind artificial neural networks started with the model of \emph{Perceptron} introduced by \citet{mcculloch1943logical}. It is a simple model which transforms a vector of inputs $s$ to an output value $y$. 

\begin{figure}[h]
  \centering
  \includegraphics[width=0.7\textwidth]{img/perceptron.pdf}    
  \caption{Perceptron. Notation: $x$ is the \emph{input vector} where always $x_0=1$, $w_{0k}$ is the \emph{weight} vector, $\Sigma$ is the \emph{summing} junction, $\eta_k$ is the \emph{net input}, $\phi$ is the \emph{activation function}, $\theta_k$ is the \emph{treshold}, $y_k$ is the \emph{output} and $b_k$ is the \emph{bias}.} 
  \label{fig:perceptron}
\end{figure}

The whole transformation of the input vector to the output activation could be written as follows: 
\begin{equation}
\label{eq:perceptron} 
y_k =
\left\{
	\begin{array}{ll}
		0 & \mbox{if } \phi(\sum_{i=0}^N x_iw_{ik}) < \theta_k \\
		1 & \mbox{otherwise}
	\end{array}
\right.
\end{equation} 

Equation~\ref{eq:perceptron} describes a simple \emph{binary treshold perceptron}. One could observe that the binary perceptron divides the vector space $\mathbb{R}^N$ by a $(n-1)$--dimensional hyperplane. This behaviour was studied by \citet{rosenblatt1958perceptron}. Now we see the importance of bias which is the absolute term in the equation of the hyperplane. 

\paragraph{Continuous perceptron.}
We put additional constraints for the activation function $\phi \mathbb{R} \mapsto (0,1)$: 
\begin{enumerate} 
\item It is differentiable and monotonously increasing,
\item and satisfying two asymptotic conditions $t(-\infty)=0$ and $t(\infty)=1$. 
\end{enumerate} 
Usually, the transfer function is realized by the logistic function: 

\begin{equation}
\frac{1}{1 + e^{-\eta}}. 
\end{equation} 

To allow the outputs to be from range $(0,1)$ we drop the treshold and simply output $\phi(\eta_k)$. 

\paragraph{Learning.} 
The goal of a perceptron is to \emph{learn} a given set $T = \{(X^j, t_j)\}$ mappings, where $X^j$ is the input vector $(x_{j0},x_{j1}, \ldots, x_jN)$ and $t_j$ is the corresponding target. Ideally, we want from the perceptron to be able generalize for novel inputs. The goal is formalized by minimizing an error function: 

\begin{equation}
\label{eq:perceptron-error} 
E = \sum_{k=1}^{N} \frac{1}{2}(y_k-t_k)^2.
\end{equation} 

A straightforward method to achieve this is simply updateing weights according to the partial derivates of the error function: 

\begin{equation}
\label{eq:perceptron-learning} 
\frac{\partial E}{\partial w_{ik}} = (y_k - t_k)\phi'(\eta_k)x_i = (y_k - t_k)y_k(1 - y_k)x_i.
\end{equation} 

From equation~\ref{eq:perceptron-learning} we can develop a following learning algorithm: 
\begin{figure}[h]
  \centering
\begin{lstlisting}
for e from 1 to EPOCHS: 
  foreach (X^j, t_j) in T: 
    calculate y_j
    for i from 0 to N: 
      update w_{ij}
\end{lstlisting}
  \caption{Perceptron learning. Developed from equations~\ref{eq:perceptron-error} and \ref{eq:perceptron-learning} we get an \emph{weight update} rule which is applied in loop for element in $T$. TODO sth better than lstlisting} 
  \label{fig:perceptron-learning}
\end{figure}

Usually, the \emph{update rule} is written as: 
\begin{equation} 
\Delta w_{ik} = \lambda (y_k - t_k)y_k(1 - y_k)x_i,
\end{equation} 
where $\lambda$ is the \emph{learning speed}. 

 
\label{sec:perceptron} 

\subsubsection{Multilayer Feedworward Networks} 
\label{sec:theory-multilayer} 

We will define \emph{multilayer feedforward networks} as in~\citet{haykin1994neural}. First, we define a \emph{layered} neural network where neurons are organised to form layers. In the simplest version we have an \emph{input layer} of source nodes and an \emph{output layer} which is formed by aforementioned perceptrons. In other words this is a \emph{feedforward} or \emph{acyclic} type of network as the \emph{activation}, i.e. outputs of the neurons are computed from the input to the output layer and never \emph{backwards}. 

\begin{figure}[H]
  \centering
  \includegraphics[width=0.5\textwidth]{img/multilayer.pdf}    
  \caption{Fully connected feedforward \emph{multilayer} network with one \emph{hidden} layer. } 
  \label{fig:multilayer}
\end{figure}

Multilayer neural network has one or more \emph{hidden layers} in addition to the input and ouput layer as shown on figure~\ref{fig:multilayer}. The source nodes supply the activation pattern, i.e. input vector, which is applied to next layer of neurons. The output signal of the hidden layer is used as the input for the output layer. As shown by~\citet{cybenko1989approximation} the three layer network is an universal approximator of continuous functions on compact subsets of $\mathbb{R}^n$.

There exists several methods for training multilayer networks. First, we will describe the most common backpropagation in~(\ref{sec:models-bp}) and then methods related to our work such as CHL~(\ref{sec:models-chl}), GeneRec~(\ref{sec:models-generec}) and BAL~(\ref{sec:models-bal}). 


\subsubsection{Recurrent networks}
\label{sec:theory-recurrent} 

In \emph{recurrent} neural networks also cycles of connections are allowed. This arises problems with computing their activations. That means that output of a particular unit could affect its input. Therefore, the activations in general could not be computed only by one forward pass. This introduces real valued dynamic systems for computing the activations. We can observe that it holds that $\partial\eta / \partial t = 0$ for the activations of neurons in the fixed point state. There are several approaches solving these dynamic systems and deriving the learning rule~\citep{pineda1987generalization, pearlmutter1989learning, williams1989learning, elman1990finding, haykin1994neural}. 

\begin{figure}[H]
  \centering
  \includegraphics[width=0.4\textwidth]{img/models-recurrent.pdf}    
  \caption{Simple recurrent network proposed by~\citet{elman1990finding}. Taken from~\citet{haykin1994neural}.} 
  \label{fig:theory-recurrent}
\end{figure}

An \emph{iterative method} is used by~\citet{movellan1990contrastive} for computing activations. In the first step the input neurons have activations equal to the input vector and the other neurons have zero activation. In the next steps activations from the last step are used to compute activation in this step as shown in equation~(\ref{eq:theory-recurrent-activation}): 
\begin{equation}
  \label{eq:theory-recurrent-activation} 
  \eta_i(t+1) = \phi\left(\sum_j w_{ji}\eta_i(t)\right) 
\end{equation}
This rule is iterated while the activations are not settled. For particullar symmetric networks it could be proved that activations will converge~\citep{o1996bio}. For more general networks a dynamic system based on rule~(\ref{eq:theory-recurrent-activation}) could be introduced. The the fixed point solution is the settled activation. \citet{movellan1990contrastive} proposes using the method of simulated annealing~\citep{kirkpatrick1983optimization,vcerny1985thermodynamical} to improve the learning rule and to avoid settling the network in a local minima. We experimented with the iterative method for a two way version of GeneRec in~Section~(\ref{sec:our-bal-recirc}). 

 

\subsubsection{Hopfield networks}
\label{sec:theory-hopfield}

\citet{hopfield1984neurons} introduced a network with arbitrary connections defined only by one weight matrix $W$. Some of the units are chosen as the \emph{input units} which have stable activations for a given input pattern. We can treat a Hopfield network as a recurrent neural network. A Hopfield network comes with a continuous energy function for which usually function~(\ref{eq:theory-hopfield-energy}) is chosen: 
\begin{equation}
  \label{eq:theory-hopfield-energy}
  E = -\frac{1}{2}\sum_i\sum_ja_iw_{ij}a_j,
\end{equation} 
where $a_i$ is the activation of the $i$-th unit. The aim of the network is to settle the activations so that $E$ settles in a global minima. Activation for the $i$-th unit is computed based on the following differential equation~\citep{hopfield1984neurons}: 
\begin{equation}
  \label{eq:theory-hopfield-activation}
  \frac{\partial a_i}{\partial t} = \alpha(-a_i + f_i(\eta_i)),
\end{equation} 
where $a^T = [a_1,\ldots,a_n]$ is the activation vector, $f_i$ is bounded, monotically increasing, differentiable activation function. \citet{hopfield1984neurons} proved for equation~(\ref{eq:theory-hopfield-activation}) that if the weights are symmetric, i.e. $w_{ij} = w_{ji}$, the activations will settle in the minimal error state defined in equation~(\ref{eq:theory-hopfield-activation}). This learning rule is typically used in \emph{interactive activation networks} studied by~\citet{grossberg1978theory} and~\citet{mcclelland1981interactive}. 

 
 

\subsubsection{Backpropagation}
\label{models-bp} 

TODO spomenut hetero-asociativne, jednosmerne, supervised 
TODO napisat ako theory example

A criticism of backpropagation is that it is neurally implausible (and hard to implement in hardware) because it requires all the connections to be used backward and it requires the units to use different input-output functions for the forward and backward passes \citet{hinton1988learning}.

The procedure repeatedly adjusts the weights of the connections in the network so as to minimize a measure of the difference between the actual output vector of the net and desired output vector \citet{rumelhart1986learning}. 

The aim is to find a powerful synaptic modification rule that will allow an arbitrarily connected neural network to develop an internal structure that is appropriate for a particular task domain \citet{rumelhart1986learning}. 

Connection within a layer or from higher to lower layers are forbidden, but connections can skip intermediate layers \citet{rumelhart1986learning}.

All units within a layer have their states set in parallel, but different layers have their states set sequentially, starting at the bottom and working upwards until the states of the output units are determined \citet{rumelhart1986learning}. 

$$x_j = \sum_i y_iw_{ji}.$$

$$y_i = \frac{1}{1 + e^{-x_i}}.$$

$$E = \frac{1}{2} \sum_c \sum_j (y_{j,c} - d_{j,c})^2,$$
where $c$ is index over cases. 

To minimize $E$ by gradient descent it is necessary to compute the partial derivate of $E$ with respect to each weight in the network \citet{rumelhart1986learning}. 

$$\partial E / \partial y_j = y_j - d_j.$$

We computed $\partial E / \partial y_j$ for any unit when $\partial E / \partial y_j$ given in the last layer. Repeating this procedure we get $\partial E / \partial y_j$ for all weights. 

Adding momentum:
$\Delta w(t) = -\epsilon \partial E/ \partial w(t) + \alpha \Delta w(t-1).$

Adding a few more connection creates extra dimensions in weight-space and these dimensions provide paths around the barriers that create poor local minima in the lower dimensional subspaces \citet{rumelhart1986learning}. 

The learning procedure, in its current form, is not a plausible model of learning in brains. 

$$\frac{\partial E}{\partial w_{ij}} = -\sum_k(t_k-o_k)w_{jk}\sigma'(\eta_j)s_i,$$
where $t_k$ is the target value, $o_k$ is the output value, $\sigma$ is the nonlinear function, $\eta_j$ is the net input and $s_i$ is the stimulus input \citet{o1996bio}.

%TODO prekreslit do IPE 
\begin{center} 
\includegraphics{img/table_bp.png} 
\citet{farkas2013bal} 
\end{center} 


\subsection{Models} %TODO change name 
\label{sec:overview-models}  
%TODO more search - I could miss a lot (see articles which reference O'Reilly 96 ..) 
%TODO more references
%TODO made up a template for each model, like 1. motivation 2. activation table 3. learning rule
%TODO write it as a comparison to BAL-like models we analysed 

In this section we briefly mention models on which our work was based on. Mainly it's the Bidirectional Activation-based Learning algorithm \ref{sec:models-bal} by \citet{farkas2013bal} and the Generalized recirculation \ref{sec:models-generec} by \citet{o1996bio}. The other two models are inspiration for the two former ones. Understanding the latter helps understanding the former. 

%\subsubsection{Boltzmann machines}
TODO \cite{ackley1985learning}

TODO: Read and cite from Hinton's original article. 

(Wiki) A Boltzmann machine is a type of stochastic recurrent neural network invented by Geoffrey Hinton and Terry Sejnowski. Boltzmann machines can be seen as the stochastic, generative counterpart of Hopfield nets. They were one of the first examples of a neural network capable of learning internal representations, and are able to represent and (given sufficient time) solve difficult combinatoric problems. If the connectivity is constrained, the learning can be made efficient enough to be useful for practical problems.

A Boltzmann machine, like a Hopfield network, is a network of units with an "energy" defined for the network. It also has binary units, but unlike Hopfield nets, Boltzmann machine units are stochastic. The global energy, $E$, in a Boltzmann machine is identical in form to that of a Hopfield network:

$$E = -\sum_{i<j} w_{ij} \, s_i \, s_j - \sum_i \theta_i \, s_i.$$

Where:
\begin{itemize}
    \item $w_{ij}$ is the connection strength between unit $j$ and unit $i$.
    \item $s_i$ is the state, $s_i \in \{0,1\}$, of unit $i$
    \item $\theta_i$ is the threshold of unit $i$.
\end{itemize}

The connections in a Boltzmann machine have two restrictions:
\begin{itemize}
    \item $w_{ii}=0\qquad \forall i$. (No unit has a connection with itself.)
    \item $w_{ij}=w_{ji}\qquad \forall i,j$. (All connections are symmetric.)
\end{itemize}

Often the weights are represented in matrix form with a symmetric matrix $W$, with zeros along the diagonal.

\paragraph{Hopfield nets.}

TODO: Read and cite from Hopfields's original article. (Storkey, Amos. "Increasing the capacity of a Hopfield network without sacrificing functionality." Artificial Neural Networks—ICANN'97 (1997): 451-456.)

(Wiki) A Hopfield network is a form of recurrent artificial neural network invented by John Hopfield. Hopfield nets serve as content-addressable memory systems with binary threshold nodes. They are guaranteed to converge to a local minimum, but convergence to a false pattern (wrong local minimum) rather than the stored pattern (expected local minimum) can occur. Hopfield networks also provide a model for understanding human memory.

\paragraph{Boltzmann distribution.}

TODO:  Landau, Lev Davidovich; and Lifshitz, Evgeny Mikhailovich (1980) [1976]. Statistical Physics. 5 (3 ed.). Oxford: Pergamon Press. ISBN 0-7506-3372-7. Translated by J.B. Sykes and M.J. Kearsley. See section 28

(Wiki) The Boltzmann distribution for the fractional number of particles $Ni / N$ occupying a set of states $i$ possessing energy $E_i$ is:

    $${N_i \over N} = {g_i e^{-E_i/(k_BT)} \over Z(T)}.$$

where $k_B$ is the Boltzmann constant, $T$ is temperature (assumed to be a well-defined quantity), $g_i$ is the degeneracy (meaning, the number of levels having energy $E_i$; sometimes, the more general \'states\' are used instead of levels, to avoid using degeneracy in the equation), $N$ is the total number of particles and $Z(T)$ is the partition function.

    $$N=\sum_i N_i,$$

    $$Z(T)=\sum_i g_i e^{-E_i/(k_BT)}. $$
  %moved to old 

%TODO equilibrium values 
\def\myover#1#2{\mathrel{\overset{\makebox[0pt]{#2}}{#1}}}
\newcommand{\mytilde}{\raise.17ex\hbox{$\scriptstyle\mathtt{\sim}$}}
\def\myequi#1{\myover{#1}{\mytilde}}

\subsubsection{Contrastive Hebbian Learning}
\label{sec:models-chl} 

The main idea of \emph{Contrastive Hebbian Learning} developed by \citet{movellan1990contrastive} is to have two activation phases in an aribtrary Hopfield network \citep{hopfield1984neurons} described in \ref{sec:theory-hopfield}. In the first phase, called \emph{minus phase} and denoted \quotes{-}, only the input vector is \emph{clamped}, i.e. activations of the clamped units as equal to the clamped values. In the second phase, called \emph{plus phase} and denoted \quotes{+}, both the input and target are clambed to the underlying network. The learning is based on the difference of these two activations. Note that CHL brings no assumptions about the structure of the underlying network and therefore it has no layers in general. 

As mentioned previously CHL is based on Hopfield networks. Therefore it has an energy function $J$ which is based on the Helmholtz free energy function $F$ \citep{hinton1989deterministic}:
\begin{equation}
  \label{eq:models-chl-helmholtz}
  F = -\frac{1}{2}\sum_i\sum_ja_iw_{ij}a_j + \sum_i \int_{f(0)}^{a_i} f_i^{-1}(a)da
\end{equation} 
where $-\frac{1}{2}\sum_i\sum_ja_iw_{ij}a_j$ is the Hopfield energy function~\ref{eq:theory-hopfield-energy}. The \emph{contrastive} error function $J$ is defined as: 
\begin{equation}
  \label{eq:models-chl-energy}
  J = \myequi{F^{+}} - \myequi{F^{-}}
\end{equation} 
where $\myequi{F^{+}}$ and $\myequi{F^{-}}$ respectively are the values of the energy functions at equilibrium states for the plus and the minus phases. 

Based on the contrastive energy function~\ref{eq:models-chl-energy} a learning rule is derived by \citet{movellan1990contrastive}: 
\begin{equation}
  \label{eq:models-chl-learning-rule}
  \Delta w_{ij} = \myequi{a_i^{+}}\myequi{a_j^{+}} - \myequi{a_i^{-}}\myequi{a_j^{-}}
\end{equation}
where $\myequi{a_i}$ and $\myequi{a_j}$ denote the equilibrium state activations of the $i$--th and $j$--th unit. It could be shown that the learning rule~\ref{eq:models-chl-learning-rule} decreases the energy function \ref{eq:models-chl-energy} \citep{movellan1990contrastive}. Moreover it could be shown that the CHL learning rule is equivalent to Backpropagation learning rule in terms of computability while it is biologically more plausible \citep{o1996bio, xie2003equivalence}. 

   


%\subsubsection{Almeida-Pineda Algorithm}

Pineda first applied the backpropagation rule to recurrent networks. Recurrent networks can have cycles and they contain no layers. Subset $\Omega$ of the units is treated as the output layer. So the Almeida-Pineda network is a generealization of the backpropagation network. 

As the network is recurrent it can remember information and can be used as CAM.
%TODO citation for CAM 

\includegraphics[width=8px]{img/recurrent.png}

\paragraph{My notes.} 
So the Pinedas NS is a dynamical system which should converge and converges for forward/backward nets (quite trivially). 

The delta rule is constructed to move the state towards the fixed point which is defined so that the difference on output neurons is equal to zero (values on other neurons are chosen arbitrary?). 

First Pineda derives an exact learning rule which requires computation of inverse matrix what is both bad for implementation and biological plausibility. So it is simplified to associate dynamical system (I don't understand how) which converges if the original dynamical system converges (proved by Almeida). 

It is more natural as all neurons are equivalent by construction. The advantage is exploited in hardware computation as it treats differential equations more naturally can be solved by analog computers. 

%TODO: Compare this conclusion with O'Reillys. 


TODO: Understand the basics of differential equations to enhance the intuition about learning rules. 

TODO: Study the Lapedes and Farber: master / slave NS. 

\paragraph{Citations from the article.}

Nevertheless it has been applied to recurrent networks by taking advanage of the fact that for every recurrent network there exists an equivalent feedforward network (for a finite time) \cite{pineda1987generalization}.

Hopfield's equations are globally asymptotically stable if $w$ is symmetric and has zeros along the diagonal \cite{pineda1987generalization}.

$$y_r = \beta f_r^,(u_r)\sum_k J_k(L^{-1})_{kr},$$
where 
$$L_{ij} = \alpha \delta_{ij} - \beta f_i^,(u_i)w_{ij},$$
and where $\delta_{ij}$ is the Kronecker $\delta$ symbol and $J_i = t_t - x_i$ if $i \in \Omega$ and $J_i = 0$ otherwise \cite{pineda1987generalization}. 

Then the exact learning rule is 
$$dw_{rs}/dt = \gamma y_r x_s.$$
Rhis exact learning rule needs matrix inversion to calculate the error signals $y_k$. Direct matrix inversions are necessarily nonlocal calculations and therefore this learning algorithm is not suitable for implementation as a neural network \cite{pineda1987generalization}. 

TODO: Understand the associated dynamical system and the new learning rule (page 3/4). 

\paragraph{O'Reillys conclusion}
\cite{o1996bio}
The activation states in AP are updated according to a discrete-time approximation of the following dif-
ferential equation, which is integrated over time with respect to the net input terms :

$$\frac{d\eta_j}{d_t} = -\eta_j + \sum w_{ij} \sigma(\eta_i).$$

This equation can be iteratively applied until the network settles into a stable equilibrium state (i.e., until the
change in activation state goes below a small threshold value), which it will provably do if the weights are
symmetric (Hopfield, 1984), and often even if they are not (Galland \& Hinton, 1991).

\includegraphics[width=10cm]{img/table_ap.png}
  %moved to old 

\subsubsection{Recirculation algorithm}
\label{models-recirc} 

TODO image of the algorithm 
TODO shorten and rewrite the text 
TODO spomenut ze unsupervised 

We will outline the main ideas used for this Recirculation algorithm. They motivation is in an autoencoder (or data-compression) and they also mention the PCA algorithm. 

First, there are just two layers - visible and hidden which are connected by directed weights (most of time we assume these weights as symmetric). 

Second, we operate in discrete time. Each learning phase has four steps. In first two the input is propagated to hidden layer. In second two the desired output is propagated to hidden layer. The learning on visible layer occurs in second step when the desired output meets the reconstructed input. Similary, the learning on hidden layer occurs in step 3.

Third, the learning rules for both layers is almost identical (we can non-formally write that they are symmetric). It is interesting that the learning rule uses no exact ifnormation about the non-linear function (usually logistics), but only assumes that it has bounded derivate. The learning rule is an approximation of the learning rule derived from the difference equations for error correction. 

Authors study the case of asymetric weights. They do not provide a proof of convergence but they provide some intutition why it should work. Also they link the work of Ballard who experimented with connecting (merging) several closed loops so that hidden units of closed loops can be input units of other closed loops of recirculation.

\paragraph{From the original article.}
Instead of using a separate group of units for the input and output we use the very same group of \textit{visible} units, so the input vector is the initial state of this group and the output vector is the state after information has passed around the loop. The difference between the activity of a visible unit before and after sending activity around the loop is the derivative of the squared reconstruction error \citet{hinton1988learning}.

On the first pass, the original visible vector is passed around the loop, and on the second pass ana average of the original vector and the reconstructed vector is passed around the loop. The learning procedure changes each weight by an amount proportional to the product of the \textit{presynaptic} activity and the \textit{difference} in the post-synaptic activity on the two passes  \citet{hinton1988learning}.

$$\Delta w_{ij} = \epsilon y_j(1)[y_i(0)-y_i(2)],$$
$$\Delta w_{ji} = \epsilon y_i(2)[y_j(1)-y_j(3)],$$
where $y_i(t)$ is the activation value of unit $i$ in time $t$.

With approximation we can derive the following learning rule
$$\frac{\partial E}{\partial w_{ji}} \approx \frac{1}{1-\lambda}y_i(2)[y_j(3)-y_j(1)].$$
An interesting property of this equation is that it does not contain a term for the gradient\footnote{We assume that it is differentiable, it is monotonous and has bounded derivate.} of the input-output function of unit $j$ so recirculation learning can be applied even when unit $j$ uses an unknown non-linearity \citet{hinton1988learning}. 

%TODO: Understand why exactly we assume the linearity of the visible layer.

%TODO: Go through the derivation of the learning rule to understand why it is working as the gradient descent. 

\paragraph{O'Reillys conclusion.}

$$\frac{\partial E}{\partial h_j} = - (\sum_k t_k w_{jk} - \sum_k o_k w_{jk})$$
Thus, instead of having a separate error-backpropagation phase to communicate error signals, one can think in terms of standard activation propagation occuting via reciprocal (and symmetric) weights that come from the ouput units to the hidden units \citet{o1996bio}. 

\includegraphics[width=15px]{img/recirculation.png}


%\input{models-deep} %moved to old 

\subsubsection{GeneRec}
\label{models-generec} 

\paragraph{Introduction - version 1} 
The algorithm GeneRec (Generic Recirculation), developed by O’Reilly \citet{o1996bio} based on back-propagation, is argued to be a more biologically plausible supervised learning algorithm: learning happens through synaptic weight modifications using only local information available in synapses. In summary, GeneRec is a generalized version of the recirculation algorithm \citet{hinton1988learning}, which overcomes the limitations of the earlier algorithm (ex.: back-propagation) by using a generic recurrent network with sigmoidal units that can learn arbitrary input and output mappings. GeneRec employs two phases: \emph{minus} and \emph{plus} \citet{da2011advances}. 

\paragraph{Introduction - version 2} 
Learning is done by the Generalized Recirculation
(GeneRec) algorithm, which is argued to be a more bio-
logically plausible form of learning, developed by O’Reilly
[9]. One motivation of the development of GeneRec was
the treatment of typical problems encountered in the
backpropagation algorithm when working with bidirec-
tional networks. There is propagation of two signals in
GeneRec: the expectation of the network (called \emph{minus}
phase) and the training signal (also called \emph{plus} phase).
\citet{schneider2009application} 

\begin{table}
  \centering
  \caption{Equilibrium network variables in GeneRec model \citet{farkas2013bal}.}
  \label{tab:gr-states}
  \begin{tabular}{|cccc|}
    \hline
    Layer & Phase & Net Input & Activation\\
    \hline
    Input (s)    & $-$ & - & $s_i$ = stimulus input\\
    \hline
    Hidden (h)   & $-$ & \hspace{0.3cm}$\eta^{-}_j = \sum_i w_{ij}s_i + \sum_k w_{kj}o^{-}_k$\hspace{0.3cm} &
    $h^{-}_j = \sigma(\eta^{-}_j)$\hspace{0.3cm}\\
          &  +  & $\eta^{+}_j = \sum_{i}w_{ij}s_i + \sum_k w_{kj}o^{+}_k$ & $h^{+}_{j} = \sigma(\eta^{+}_j)$ \\
    \hline
    Output (o) & $-$ & $\eta^{-}_k = \sum_j w_{jk}h_j$ & $o^{-}_k = \sigma(\eta^{-}_k)$\\
           &  +  & - & $o^{+}_k$ = target output \\
    \hline
  \end{tabular}
\end{table}

The basic weight update rule in GeneRec is:
\begin{equation}
  \Delta w_{pq} = \lambda \ a^{-}_p(a^{+}_q - a^{-}_q)
\label{eq:generec}
\end{equation}
where $a^{-}_p$ denotes the presynaptic and $a^{-}_q$ denotes the postsynaptic unit activation in minus phase, $a^{+}_p$ is the presynaptic activation from plus phase (in output-to-hidden direction) and $\lambda$ denotes the learning rate. The learning rule given in Eq.~\ref{eq:generec} is applied to both input-hidden and hidden-output weights.  Due to the lack of space, the reader is left to consult the original paper \citet{o1996bio} regarding the underlying math behind the derivation of the GeneRec learning rule.

\paragraph{Minus phase.} When units xi are presented to the input layer A, there is the propagation of this stimulus to the hidden layer B (bottom-up propagation) (figure 1). At the same time, the previous output ok propagates from the output layer C to the hidden layer B (top- down propagation) (figure 2). Then the hidden activation \emph{minus} - minus phase - ($h^-$ ) is generated (sum of bottom-up and top-down propagations). The activation function $\sigma$ is sigmoid. Equation 1 shows the hidden activation calculus for one hidden unit j. wij are the synaptic weights from the input layer to the hidden layer and wjk are the synaptic weights from the hidden layer to the output layer (which are the same as the weights from the output layer to the hidden layer, because reciprocal weights are symmetric in GeneRec, that is, wjk = wkj \citet{o1996bio}). ok(t-1) is the previous output (output on time t - 1) \citet{orru2008sabio}.

$$h_j^- = \sigma \left(\sum_{i=0}^A w_{ij} \cdot x_i + \sum_{k=1}^C w_{jk} \cdot o_k(t-1)\right)$$

Finally, the real output ok(t) is generated through the propagation of the \emph{minus} layer activation to the output layer (figure 3), shown for one output unit k by equation 2 \citet{o1996bio}. Notice that the architecture employed is bi-directional. Recall that ok (t) (the current output on time t) is used in order to differentiate it from ok (t - 1) (the previous output on time t - 1).

\begin{center} 
\includegraphics{img/generec_minus_phase.png} \citet{orru2008sabio} 
\end{center} 

\paragraph{Plus phase.} Units xi are presented again to the input layer A; there is the propagation of this stimulus to the hidden layer B (bottom-up propagation) (figure 4). At the same time, the desired output yk propagates from the output layer C to the hidden layer B (top- down propagation) (figure 5). Then the hidden activation + \emph{plus} - plus phase - (hj) is generated, summin bottom-up and top-down propagations (equation 3) \citet{o1996bio}, \citet{orru2008sabio}.

$$h_j^+ = \sigma\left( \sum_{i=0}^A w_{ij} \cdot x_i + \sum_{k=1}^C w_{jk} y_k \right)$$

\begin{center} 
\includegraphics{img/generec_plus_phase.png} \citet{orru2008sabio} 
\end{center} 

In order to make learning possible, synaptic weights $w$ are updated, based on h-, j  $h^+_j$, $o_k$ , $y_k$, $x_i$, and the learning rate $\eta$ (equations 4 and 5).

%TODO equation labels with intext references 
$$\Delta w_{jk} = \eta(y_k - o_k(t)) h^-_j $$

$$\Delta w_{ij} = \eta(h^+_j - h^-_j) x_i$$

Finally, O’Reilly \citet{o1998six} suggests, unlike backpropagation, that the teaching signal is just another state of “experience” in the network, that is, in GeneRec algorithm the teaching signal is exactly the “top-down” activation in the plus-phase.


%TODO reformulate 
It was recently shown that backpropagation can be implemented in a more biologically plausible fashion using bidirectional activation propagation in an interactive network using the GeneRec algorithm \citet{o1996bio}, which is a generalization of the recirculation algorithm \citet{hinton1988learning}. In GeneRec, error information is propagated as two separate terms via standard activation propagation mechanisms in interactive networks, and the difference between these terms (which is the error signal) can be plausibly computed using the synaptic modification mechanisms underlying long term potentiation and depression (LTP/LTD). Versions of the GeneRec algorithm are equivalent to the other known ways of implementing powerful error-driven learning using interactive activation propagation instead of direct error propagation (e.g., the deterministic Boltzmann machine \citet{hinton1989deterministic} and Contrastive Hebbian Learning \citet{movellan1990contrastive}). Thus, several different approaches converge on the idea that the way to perform error-driven learning in a more biologically plausible manner is to use interactive networks, where error signals are communicated via top-down activation propagation \citet{o2001generalization}.

Copy from \citet{da2011advances} 


%TODO try out the following rules: 
Hebbian learning is performed using a Conditional Principal Components Analysis (CPCA) algorithm with a correction factor for sparse expected activity levels [3]. The error-driven learning is achieved with GeneRec; the output is computed in two phases – an expectation phase where the network's actual output is produced and an outcome phase where the target output is experienced – as a difference of a pre- and postsynaptic activation product across these two phases. Hebbian weights are adjusted according to the following formula.

equation image	(2)
while error-driven learning uses the following equation.

equation image	(3)
where xi is the input of neuron i, yj is the output of neuron j, and wij is the connection weight between neurons i and j. The “+” and “– superscripts refer to plus and minus phases of the GeneRec algorithm \citet{nawrocki2012monitoring}.

Some improvements \citet{da2008biological}. 

To assess the effect of interactivity, two different networks were compared on the combinatorial generalization task, a standard feedforward backpropagation network, and an interactive GeneRec network using the symmetric, midpoint variation learning rule which is equivalent to contrastive Hebbian learning (CHL) or a deterministic Boltzmann machine (DBM) \citet{o1996bio}, \citet{o2001generalization}. 
 
\footnote{TODO: How to cope with biases (which are not symmetric)? We haven't found how GeneRec uses the Bias neuron.} 


%==================== 12.   Background overview (optional) ======
\subsection{Bidirectional Activation-based Learning algorithm} 
\label{models-bal} 
% If your work builds on top of an existing one, this is the place to describe the existing work in more detail, pointing out the parts that you extend or improve and why you extend or improve these parts.

Design of Bidirectional Activation-based Learning algorithm (BAL) by \citet{farkas2013bal} is motivated by the biological plausibility of GeneRec. BAL inherits the learning rule of GeneRec \ref{eq:models-generec-learning-rule} and also the two phases. But unlike GeneRec, BAL aims to learn bidirectional mapping between inputs and outputs and for this purpose it uses four weights $W^{IH}$, $W^{HO}$, $W^{OH}$ and $W^{HI}$. The design of BAL is symmetric as shown in table~\ref{tab:bal-activation} and thus we avoid calling inputs, outpus, minus phase or plus phase. We rather choose \emph{forward} and \emph{backward} which could be interchanged. Note that the forward activations are denoted as $a^{\rm F}$ and backward activations as $a^{\rm B}$. 

\begin{table}
  \label{tab:models-bal-activation}
  \centering
  \begin{tabular}{|cccl|}
    \hline
    Layer & Phase & Net Input & Activation\\
    \hline
    \Bx & F & - & $x^{\rm F}_i$ = stimulus\\ [1ex]
    \Bh & F & \hspace{0.3cm}$\eta^{\rm F}_j = \sum_i w_{ij}^{IH}x^{F}_i$\hspace{0.3cm} & $h^{\rm F}_j = \sigma(\eta^{\rm F}_j)$\hspace{0.3cm}\\ [1ex]
    \By & F & $\eta^{\rm F}_k = \sum_j w_{jk}^{HO}h^{F}_j$ & $y^{\rm F}_k = \sigma(\eta^{\rm F}_k)$\\ [1ex]
    \hline
    \By & B & - & $y^{\rm B}_k$ = stimulus\\ [1ex]
    \Bh & B & $\eta^{\rm B}_j = \sum_k w_{kj}^{OH}y^{\rm B}_k$ & $h^{\rm B}_j = \sigma(\eta^{\rm B}_j)$\\ [1ex]
    \Bx & B  & $\eta^{\rm B}_i = \sum_j w_{ji}^{HI}h^{\rm B}_j$ & $x^{\rm B}_i = \sigma(\eta^{\rm B}_i)$\\
    \hline
  \end{tabular}
  \caption{Activation phases and states in BAL \citep{farkas2013bal}. Where \Bx is the first activation layer, i.e. \emph{front layer}, \By is the third activation layer, i.e. \emph{back layer}, $F$ means \emph{forward pass} and $B$ means \emph{backward pass}. Layers \Bx and \By are \emph{visible} and layer \By is hidden. Note that all non--stimulus units have learnable biases and their weights are updated in a same way as regular weights. } 
\end{table}

In the first phase called \emph{forward pass} the \emph{forward stimulus} is clamped and forward activations are computed. In the same way, in the second phase called \emph{backward pass} the \emph{backward stimulus} is clamped and backward activations are computed. We can imagine the backward pass as a reconstruction of the target pattern for the forward pass. For the learning rule the \emph{difference} between the forward pass and the backward pass is used: 
\begin{equation}
  \label{eq:models-bal-learning-rule-forward}
  \Delta w_{ij}^{\rm F} = \lambda \ a_i^{\rm F}(a_j^{\rm B} - a_j^{\rm F}),
\end{equation}
and for completeness we also provide the backward learning rule which is same as the forward learning rule~\ref{eq:models-bal-learning-rule-forward}: 
\begin{equation}
  \label{eq:models-bal-learning-rule-backward}
  \Delta w_{ij}^{\rm B} = \lambda \ a_i^{\rm B}(a_j^{\rm F} - a_j^{\rm B}). 
\end{equation}
Both forward~\ref{eq:models-bal-learning-rule-forward} and backward~\ref{eq:models-bal-learning-rule-backward} learning rules are same as the basic GeneRec learning rule~\ref{eq:models-generec-learning-rule}. We experimented with different learning learning rules \ref{sec:our-learning-rules}. 

 


 
