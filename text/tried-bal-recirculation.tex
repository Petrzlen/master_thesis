\subsubsection{Recirculation Bidirectional Activation-based Learning algorithm} 
TODO!: Analysis of fluctuation. 
TODO!: Try different lambda on layers. 
About 33\% success (lambda 0.3). 

\paragraph{Overview} 
(5th of March) 
How to generalize GeneRec to both ways? 
Experimenting with:
\begin{itemize} 
\item  a) classic generec - use only 3 matrices (which should work best for symmetric) 
\item  b) bothwards = f(hidden\_net\_from\_input + hidden\_net\_from\_output) 
\item  c) using GeneRec distinctly for forward and backward and combine them 
\end{itemize} 

\paragraph{Recirculation step} 
If no stationary point is found after MAX\_ITERATION then we set the result to avarage of the last two activations. This lead to 80\% less fluctuation. But still occured: 
\begin{itemize}
\item Big fluctuation: 0.9591299483462391
\item Not enough iterations: 20
\item Even if MAX\_ITERATION = 200 then max fluctuation still could be arbitrary and oscilating.
\end{itemize} 

\paragraph{Interesting} 
\begin{itemize} 
  \item oscilation in iteration could occur randomly (just in some epochs and it will completely change the network) 
  \item when using averages, it's less probable that a fluctuation will occur 
  \item setting MAX\_iteration much higher doesn't affect performance. About 50 is enough but 20 is not. 
\end{itemize} 
  
\paragraph{Conclusion} 
Symmetric weights haven't helped BAL (35\%) - no param selection
Using bothward GeneRec (60\%) - no param selection 
  Forgot to symmetric init -> no perceivable change 
In both cases (almost) no fluctuation 

Non-symmetric case, fluctuation in about 1/5 cases 
  About 30\% success 
  About 3-33 iterations needed to settle, very network dependent 
  About 50 - 5000 epochs needed 

IDEA: maybe bad implementation of backward
      using iterative activation has almost no reason (bothward is the generec idea) 
