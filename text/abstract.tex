%Abstract is very important part of the thesis. It will be most read by people and should be written with a great care. The abstract should mention:
% 1) About the problem you want to solve
% 2) About your solution – how you solve the problem
% 3) Highlights about how good is your solution (e.g. achieves 70\% better performance) referring to the results you obtained in your experiments (e.g. achieves 70\% better performance).
% 4) Possible impacts of your work into the field (e.g. “The proposed solution can be used to offload the CPU by executing data parallel computation intensive code on GPUs and thus obtaining additional Speedup for no cost”).

\section*{Abstract}

% 1) About the problem you want to solve
In our work, we used computational simulations to analyse supervised artificial neural networks based on the Generalized recirculation algorithm (GeneRec) by~\citet{o1996bio} and the Bidirectional Activation-based Learning algorithm (BAL) by~\citet{farkas2013bal}. The main idea of both algorithms is to update weights based on the difference between forward and backward propagation of neuron activations rather than based on error backpropagation (BP) between layers, which is considered biologically implausible. However, both algorithms struggle to learn low dimensional mappings which could be easily learned by BP. The aim of this work is to fill this gap. 

% 2) About your solution – how you solve the problem
Several modifications of BAL are proposed and after systematic analysis a Two learning rates (TLR) version is introduced. TLR uses different learning rates for different weight matrices. The simulations prove increase in success rate and show smooth relation between success and learning rates. For the networks with highest success rate the two learning rates can differ even by $10^6$. Further the idea of TLR is applied to GeneRec. Finally, additional experiments for momentum, weight initialization, hidden activations and dynamic learning rate are analysed. 

% 4) Possible impacts of your work into the field (e.g. “The proposed solution can be used to of
We believe that using the idea of TLR could lead to performance increase in other artificial neural network models as wlll, and even multi-layered networks. Intuitively, an increase in success rate could be achieved by generalizing the idea of TLR to additional parameters, such as momentum or weight initialization. Further experiments are outlined. 

\begin{flushleft}
  \textbf{Keywords:} supervised learning, artificial neural network, heteroassociative mapping, dynamic learning rate, activation based learning. 
\end{flushleft}

%keywords={ Internet; TCP streams; Tor network;}

