%This section should contain a little about everything. Introduction should be an overview of the contents of your thesis. The introduction should contain:

\section*{Introduction}
\label{sec:introduction} 
\markboth{INTRODUCTION}{}    
\addcontentsline{toc}{section}{Introduction}

% 1) Information/introduction about the topic of your research (e.g. what you will talk about in your thesis – Compilers, CPUs, optimizations etc.)

% 2) The practical and theoretical value of the topic (how and why this topic is important) 
The field of artificial neural networks (ANN) gets lots of attention nowadays. ANNs are interesting for both psychologists and computer scientists. For psychologists they provide a simulation environment of the human brain which could be used to prove their hypotheses. For computer scientists ANNs are a general model which could be used to solve a broad range of practical problems. 

% ==================== 10. Motivation (or Problem Definition and Proposed Solution) =====
% In this chapter you have to concisely explain the problem that you want to solve and the goal of your solution. 

% 3) The motivation for your thesis (State with at least one sentence the problem you attack in your research work, why did you choose this problem and how it is interesting.  State with at least one sentence your solution for the problem).

% 4) If you are basing your work on currently existing work, mention it here.
% ==================== 10. Motivation (or tProblem Definition and Proposed Solution) =====
%In this chapter you have to concisely explain the problem that you want to solve and the goal of your solution. This part should contain:

% 1) Detailed analysis of the problem and its limitations (e.g. what is the bottleneck and difficulties).

%\subsection*{Motivation}
\label{sec:motivation} 
% 1) Detailed analysis of the problem and its limitations (e.g. what is the bottleneck and difficulties).
We choose to analyse the Bidirectional Activation-based Learning algorithm (BAL) recently introduced by~\citet{farkas2013bal}. The main reasons why we have chosen BAL over standard models are its simplicity, \emph{bidirectionality} and \emph{biological plausibility}. The term biological plausibility is stated by six principles by~\citet{hinton1988learning}. One of these principles is the \emph{bidirectional activation propagation} is achieved in BAL by using backward representations for learning the forward representations and vice versa. 

% motivation.3) You should clearly state and explain your goal and objectives. You should provide analytical study (mathematical model) of your solution (What is the upper and lower bound of your performance or improvements). You should also mention about the qualitative benefit of your solution such as easy programming etc..

% 5) Mention the limitations of your solution (design and implementation – e.g. applies for real time systems, has error factor 25%)

% 6) Include information about your key results – e.g. we improve the performance with 70% in the general keys.

%\subsection*{Goals and Results}
Although BAL performs well on high dimensional tasks, it has problems to learn low dimensional tasks with 100\% reliability, while BP is able to learn them. Therefore, our primary goal was to find reasons for this performance gap and use them to derive a modification of BAL which will perform comparably to BP. In our work, we were able to follow this process and find reasons which were then used for deriving the Two learning rate model (TLR). TLR uses different learning rates for different weight matrices and shows some counter intuitive behaviour. It performs comparable to BP if it is initialized correctly. 

% 7) Finish the chapter with an overview about the contents of your thesis.la, ktory temu trocha chape, zavedenie zakladnej notacie 
%\subsection*{Overview of contents}

Our work starts with an overview of ANNs in Chapter~\ref{sec:overview} which consists of necessary preliminaries and related models. 
In Chapter~\ref{sec:simulations} we describe our simulations, experiments and datasets used through our work. 
We finish our work with Chapter~\ref{sec:results} which contains simulation results aimed at TLR, which proved to be the most successful modification of BAL. Finally we conclude our results and outline future experiments in Section~\ref{sec:conclusion}.  

%\begin{itemize} 
%\item   co je problem
%\item   preco zaujimave,
%\item   co je zname,
%\item   my sme urobili toto
%\end{itemize} 


