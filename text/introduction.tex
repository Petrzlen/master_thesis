%This section should contain a little about everything. Introduction should be an overview of the contents of your thesis. The introduction should contain:

% 1) Information/introduction about the topic of your research (e.g. what you will talk about in your thesis – Compilers, CPUs, optimizations etc.)

% 2) The practical and theoretical value of the topic (how and why this topic is important)

% 3) The motivation for your thesis (State with at least one sentence the problem you attack in your research work, why did you choose this problem and how it is interesting.  State with at least one sentence your solution for the problem).

% 4) If you are basing your work on currently existing work, mention it here.

% 5) Mention the limitations of your solution (design and implementation – e.g. applies for real time systems, has error factor 25%)

% 6) Include information about your key results – e.g. we improve the performance with 70% in the general keys.

% 7) Finish the chapter with an overview about the contents of your thesis.

\section*{Introduction}
\label{sec:introduction} 
\markboth{INTRODUCTION}{}    
\addcontentsline{toc}{section}{Introduction}

%Neural networks are a general method in Machine Learning used when everything other fails. 

(TODO)

UVOD - pre citatela, ktory temu trocha chape, zavedenie zakladnej notacie 
\begin{itemize} 
\item   co je problem
\item   preco zaujimave,
\item   co je zname,
\item   my sme urobili toto
\end{itemize} 

We analyse and improve BAL~(\ref{sec:models-bal}) in terms of success and epoch while being inspired by Generec~(\ref{sec:models-generec}), CHL~(\ref{sec:models-chl}) and others. 

