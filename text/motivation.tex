% ==================== 10. Motivation (or Problem Definition and Proposed Solution) =====
%In this chapter you have to concisely explain the problem that you want to solve and the goal of your solution. This part should contain:

% 1) Detailed analysis of the problem and its limitations (e.g. what is the bottleneck and difficulties).

% 2) Your research methods – how did you identify these problems (e.g. tools used)?

% 3) You should clearly state and explain your goal and objectives. You should provide analytical study (mathematical model) of your solution (What is the upper and lower bound of your performance or improvements). You should also mention about the qualitative benefit of your solution such as easy programming etc..

% If necessary you may divide this chapter in sections and subsections.

\subsection*{Motivation}
\label{sec:motivation} 
% 1) Detailed analysis of the problem and its limitations (e.g. what is the bottleneck and difficulties).
In this thesis we analyse the Bidirectional Activation-based Learning algorithm (BAL) designed by \citet{farkas2013bal}. The two main advantages of BAL over standard models such as Backpropagation are \emph{bidirectional activation propagation} and \emph{biological plausibility.}

\paragraph{Bidirectional activation propagation.} In neural network models such as Backpropagation (\ref{sec:models-bp}) only the \emph{forward} mapping is learned. But in BAL, also the \emph{backward} mapping is learned while learning the forward mapping. This could be used to solve various problems, such as mapping of perception and action in robotics. Moreover, the bidirectionality of the architecture is necessary to simulate a biological electrical synapse, which can be bidirectional \citep{kandel1995essentials}, \citep{rosa2002biologically} and there is evidence that the cerebral cortex is connected in a bidirectional way and distributed representations prevail in it \citep{o2000computational}, \citep{da2011advances}. 

\paragraph{Biological plausibility.} We believe that inspiration by natural neural networks will bring results in the long run. Therefore we follow the six principles of biological plausibility stated by \citet{hinton1988learning}. The main principle is \emph{bidirectional activation propagation} mentioned above. The second principle \emph{distributed representations} states that \enquote{
  A distributed representation uses multiple active neuron like processing units to encode information (as opposed to a single unit, localist representation), and the same unit can participate in multiple representations. Each unit in a distributed representation can be thought of as representing a single feature, with information being encoded by particular combinations of such features \citep[pp.~ 456]{o1998six}
}. Therefore we aim to have as little global information as possible. Biologically plausible models are used to simulate BioAnt by \citet{schneider2009application} and cells by \citet{nawrocki2012monitoring}. 

% 3) You should clearly state and explain your goal and objectives. You should provide analytical study (mathematical model) of your solution (What is the upper and lower bound of your performance or improvements). You should also mention about the qualitative benefit of your solution such as easy programming etc..
\paragraph{Goal.}
While BAL performs well on high dimensional tasks, it was not able to learn low dimensional tasks with 100\% reliability. Other models such as Backpropagation \citep{rumelhart1986learning} learn these tasks with 100\% success. We wanted to find out the reasons for this performance gap and hoped that by solving it we could increase the performance of BAL also for the high dimensional tasks. 



