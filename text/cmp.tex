%V ďalšej časti prezentujte vlastný prínos a vlastné výsledky porovnajte s výsledkami iných. Charakterizujte použité metódy.
%Vyhýbajte sa používaniu žargónu.
%Používajte starú múdrosť: 1 obrázok je viac než 1000 slov.

\section{Comparison of Models} 
\label{sec:results-comparison}

TODO: Introduce shortcuts and add references to descriptions. \\

\subsection{4-2-4 Encoder}
\ref{sec:datasets-auto4} 

For all our models \ref{sec:our-models} tested on the 4-2-4 encoder task~\ref{sec:datasets-auto4} the two learning rate model~\ref{sec:our-two-lambdas} had the best success rate. 

For TLR, BAL, GeneRec, BP, CHL, other learning rules
TODO: table: best parameter setting networks (success, epoch, stddev) / model \\

\subsubsection{Hidden activations.}
\ref{sec:our-hidden-activation}  

%===== TODO hidden activation timelines with commentaries (for TLR, BAL, GeneRec) 
% 2x success, 2x error (wrong settle, divergence) 

\subsection{Complex Binary Vector Associations}
\ref{sec:datasets-k3} 

%===== TODO table: best parameter setting networks with hidden.size= constant (success, epoch, stddev) / model \\

\subsubsection{Different hidden sizes} 

\subsection{Hand--written digits.}
\ref{sec:datasets-digits} 

TODO simulation + plots \ref{sec:datasets-digits} 

For TLR, BAL, GeneRec + known performers 
TODO: table: best parameter setting networks with hidden.size= constant (success, epoch, stddev) / model \\

\subsubsection{Backward representations.} 
\ref{sec:our-backward-repre} 

TODO 3x 10x backward digit representations 
