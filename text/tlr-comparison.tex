%V ďalšej časti prezentujte vlastný prínos a vlastné výsledky porovnajte s výsledkami iných. Charakterizujte použité metódy.
%Vyhýbajte sa používaniu žargónu.
%Používajte starú múdrosť: 1 obrázok je viac než 1000 slov.

\subsection{Comparison} 
\label{sec:results-comparison}

TODO: GNUPlot std bars \\  
TODO: Introduce shortcuts and add references to descriptions. \\

\subsubsection{4-2-4 Encoder}

For all our models \ref{sec:our-models} tested on the 4-2-4 encoder task~\ref{sec:datasets-auto4} the two learning rate model~\ref{sec:our-two-lambdas} had the best success rate. 

For TLR, BAL, GeneRec, BP, CHL, other learning rules
TODO: success / epochs  \\
TODO: table: best parameter setting networks (success, epoch, stddev) / model \\

\paragraph{Hidden activations.} 

TODO hidden activation timelines with commentaries (for TLR, BAL, GeneRec) 
2x success, 2x error (wrong settle, divergence) 

\subsubsection{Complex Binary Vector Associations}
\ref{sec:datasets-k3} 

For TLR, BAL, GeneRec, other learning rules
TODO: bitSucc, patSucc / epochs  \\
TODO: success / hidden layer size  \\
TODO: table: best parameter setting networks with hidden.size= constant (success, epoch, stddev) / model \\

\subsubsection{Hand--written digits.}
TODO simulation + plots \ref{sec:datasets-digits} 

For TLR, BAL, GeneRec + known performers 
TODO: table: best parameter setting networks with hidden.size= constant (success, epoch, stddev) / model \\

\paragraph{Backward representations.} 

TODO 3x 10x backward digit representations 
