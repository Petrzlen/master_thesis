
\subsubsection{Comparison} 
\label{sec:tlr-auto4-cmp} 

In the table~\ref{tab:results-cmp-auto4} we can see the comparison of the most important models which we analysed on the \emph{4-2-4 encoder} task. We achieved an improvement of BAL $patSucc^F$ from $62.7\%$ to $93.1\%$ by using two different learning rates~(\ref{sec:our-tlr}). This result was further improved to $99.86\%$ by selecting networks with candidate selection~(\ref{sec:sim-exp-candidates}). This proved that hidden distance and convexity of hidden representations are important attributes of BAL~(\ref{eq:our-fitness-function}, \ref{fig:results-tlr-auto4-epoch}). 

On the other hand, many of the analysed models compared less to BAL. We tried the alternative GeneRec learning rules~(\ref{sec:models-generec-modifications}) on BAL, calling these models \emph{BAL GeneRec Learning Rules (BAL GLR)}. There was no instance which learned the task with any other rule. Also, \emph{BAL-recirc}~(\ref{sec:our-bal-recirc}) performed worse than BAL. We experimented with \emph{momentum} in section~(\ref{sec:results-momentum}) or BAL with symmetric weights, but both without significant improvement in success rate.

\begin{table}[H] 
  \centering
    \begin{tabular}{|l|l|l|l|l|}
    \hline
    Algorithm (section)&$\lambda_h$&$\lambda_v$&$patSucc^F$ &Epochs\\ %&SEM(success) \\
    \hline
    BP~(\ref{sec:models-bp}) &2.4 &2.4 &100&60\\ %&5.1\\
    \hline
    GR~(\ref{sec:models-generec}) &0.6 &0.6 &90&418\\ %&28\\
    \hline
    GR Sym~(\ref{eq:models-generec-learning-rule-sym}) &1.4 &1.4 &56&88\\ %&2.9\\
    \hline
    GR Mid~(\ref{eq:models-generec-learning-rule-mid}) &2.4 &2.4 &92&60\\ %&3.4\\
    \hline
    CHL~(\ref{sec:models-chl}) &1.2 &1.2 &56&77\\ %&1.8\\
    \hline
    BAL~(\ref{sec:models-bal})&0.9 &0.9 &62.7& 5136.11\\ %&2.0e+08\\
    \hline
    BAL TLR~(\ref{sec:our-tlr})&0.0002  & 500&93.12&5845.01\\ %&1.52e+08\\
    \hline
    BAL TLR Can~(\ref{sec:sim-exp-candidates})&0.0002&500&99.86&150.417\\ %&5,070,000\\
    \hline
    BAL Recirc~(\ref{sec:our-bal-recirc})&0.0001&1.0&36.0&1221.6\\ %&4.31e+07\\
    \hline
    BAL GLR~(\ref{sec:models-generec-modifications})& any & 0 & 0 & N/A \\
    \hline 
    \end{tabular}
  \caption{Comparison of different models on the \emph{4-2-4 encoder} task. Results for BP, GR, GR Sym, Gr Mid and CHL are taken from~\citet{o1996bio}.} 
  \label{tab:results-cmp-auto4}
\end{table}

When we want to compare execution time based on \emph{epochs} in table~\ref{tab:results-cmp-auto4}, we must be aware of that GeneRec and BAL-recirc epochs take longer than epochs of others. This is because the recirculation step~(\ref{sec:models-generec-activation}), for which about 3--33 iterations are needed for activation to settle~(\ref{sec:generec-fluctuation}). Thus the 418 epochs of GeneRec are comparable to the 5845 epochs of TLR in terms of execution time. 
