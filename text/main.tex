%http://www.feradz.com/How_to_Write_Thesis.html
%Matici sa staruju len aby predali spravnost, Informatici aj motivaciu.

%TODO byt strucny ako sa len da
%TODO uvod & prehlad = preco je tento vysledok dolezity



\documentclass[12pt,a4paper]{article}

\usepackage[slovak]{babel}
\usepackage[utf8]{inputenc}
\usepackage{a4wide}
\usepackage{tabularx}
\usepackage{amsfonts}
\usepackage{amssymb}
\usepackage{amsmath}
\usepackage{epsfig}
\usepackage{color}
\usepackage{mathrsfs}
\usepackage{verbatim}
\usepackage{hyperref}
\usepackage{subfigure}
\usepackage{float}
\usepackage{longtable}
\usepackage{listings}
\usepackage{multicol}
\usepackage{graphicx}
\usepackage{pdfpages}
\usepackage{lastpage}
\usepackage{fancyhdr}
\usepackage{url}
\usepackage[small,bf]{caption}
\usepackage[T1]{fontenc}   

                                  
\hypersetup{%  http://www.tug.org/applications/hyperref/
    bookmarksnumbered,
    pdfstartview={FitH},
    %linkcolor=black,
    %citecolor=black,
    colorlinks=true
}

\let\stdsection\section{}
\renewcommand\section{\newpage\stdsection}

%%%%%%%%%%%%%%%%%%%%%%%%% CUSTOMIZACIA BEGIN %%%%%%%%%%%%%%%
\setlength{\textheight}{24cm}
\setlength{\textwidth}{15.5cm}
\addtolength{\voffset}{-1.2cm}
\addtolength{\hoffset}{0.0cm}
\setlength{\parindent}{0.5cm}
\setlength{\parskip}{0in}
\setlength{\headheight}{16pt}
\linespread{1.5}

\hypersetup{
    colorlinks=true,       % false: boxed links; true: colored links
%    linkcolor=black,          % color of internal links
%    citecolor=black,        % color of links to bibliography
%    urlcolor=black,           % color of external links
%    linkbordercolor=black, % 	color of frame around internal links (if colorlinks=false)
%    citebordercolor=black, %	color of frame around citations
%    urlbordercolor=black, %	color of frame around URL links
}

%\setlength{\topmargin}{-10mm}
%\setlength{\textwidth}{16truecm}
%\setlength{\textheight}{24truecm}
%\setlength{\oddsidemargin}{0mm}
%\setlength{\evensidemargin}{0mm}
%%%%%%%%%%%%%%%%%%%%%%%%% CUSTOMIZACIA END %%%%%%%%%%%%%%%

\begin{document}

% ==================== 0. Cover ========================
\setcounter{page}{1}
\pagenumbering{roman}
\thispagestyle{empty}

    \begin{center}
    \large{
        \textbf{
            UNIVERZITA KOMENSKÉHO V BRATISLAVE \\ 
            FAKULTA MATEMATIKY, FYZIKY A INFORMATIKY
        }
    }
\end{center}

\vspace{2cm}

\begin{figure}[!h]
    \centering
    \includegraphics[width=3.5cm]{img/komlogo-new}
\end{figure}

\vspace{1cm}

\begin{center}
    \large{
        \textbf{
            ANALYSIS OF THE GENERALIZED \\
            RECIRCULATION-BASED LEARNING ALGORITHM \\
            IN BIDIRECTIONAL NEURAL NETWORK \\
            \vspace{3cm}
            DIPLOMA THESIS
        }
    }
\end{center}

\vfill

\begin{multicols}{2}
    \begin{flushleft}
        \textbf{Bratislava 2014}
    \end{flushleft}
    \begin{flushright}
        \textbf{Bc. Peter CSIBA}
    \end{flushright}
\end{multicols}

    
% ==================== 1. Title ========================
%You should be very careful choosing a title for your thesis. It should exactly describe what your thesis is about. You should avoid long title because it is difficult to remember. Also the title shouldn’t be too short and mention just about a general problem or a field. Example title might be “Dynamic Thread Scheduling in Heterogeneous Multi Processors” or “Static Analysis for Unprotected Concurrent Memory Accesses” or “Detecting and Executing Parallel Code in GPUs”.
\newpage
\thispagestyle{empty}
%\addtolength{\hoffset}{4mm}
%\setlength{\oddsidemargin}{-5mm}
%\setlength{\evensidemargin}{5mm}

    \begin{center}
    \large{
        \textbf{
            UNIVERZITA KOMENSKÉHO V BRATISLAVE \\ 
            FAKULTA MATEMATIKY, FYZIKY A INFORMATIKY
        }
    }
\end{center}

\vspace{2cm}

\begin{figure}[!h]
    \centering
    \includegraphics[width=3.5cm]{img/komlogo-new}
\end{figure}

\vspace{1cm}

%TODO slovensky nazov 
\begin{center}
    \large{
        \textbf{
            ANALYSIS OF THE GENERALIZED \\
            RECIRCULATION-BASED LEARNING ALGORITHM \\
            IN BIDIRECTIONAL NEURAL NETWORK \\
            \vspace{3cm}
            DIPLOMOVÁ PRÁCA
        }
    }
\end{center}

\vfill

\begin{multicols}{2}
    \begin{flushleft}
        \textbf{Bratislava 2014}
    \end{flushleft}
    \begin{flushright}
        \textbf{Bc. Peter CSIBA}
    \end{flushright}
\end{multicols}



    
% ==================== 1+. Assignment ========================
\newpage
\thispagestyle{empty}

    
\thispagestyle{empty}
\includepdf[offset=0 -25mm]{img/zadanie-en.pdf}

\newpage
\thispagestyle{empty}
\includepdf[offset=0 -25mm]{img/zadanie-sk.pdf}


% ==================== 2. Dedication ========================
% Write to whom you dedicate you thesis if any. For example “I would like to dedicate this thesis to my mother and father…”
\newpage
%\thispagestyle{empty}
\thispagestyle{plain}

    %Write to whom you dedicate you thesis if any. For example “I would like to dedicate this thesis to my mother and father…”



% ==================== 3. Acknowledgements ========================
%It’s good to acknowledge the people who helped or participated in direct or indirect way to your thesis. Here is the place to thank to your supervisor and colleagues.
\newpage
%\thispagestyle{empty}
\thispagestyle{plain}

    % ==================== 3. Acknowledgements ========================
%It’s good to acknowledge the people who helped or participated in direct or indirect way to your thesis. Here is the place to thank to your supervisor and colleagues.

\section*{Acknowledgements}

The completition of this thesis could be not possible without my supervisor Igor Farkaš who encouraged and helped me out through the whole process. Especially, I want thank him for investing his precious time in our weekly sessions which navigated me what do next and made me work regularly. 

Second but not last, I want to thank my parents Viktor and Jarmila for their lifelong investement for providing me best education they could while often sacrifying their own comfort. 


% ==================== 4. Abstract  ========================
%Abstract is very important part of the thesis. It will be most read by people and should be written with a great care. The abstract should mention:

\newpage
%\thispagestyle{empty}

    %Abstract is very important part of the thesis. It will be most read by people and should be written with a great care. The abstract should mention:
% 1) About the problem you want to solve
% 2) About your solution – how you solve the problem
% 3) Highlights about how good is your solution (e.g. achieves 70\% better performance) referring to the results you obtained in your experiments (e.g. achieves 70\% better performance).
% 4) Possible impacts of your work into the field (e.g. “The proposed solution can be used to offload the CPU by executing data parallel computation intensive code on GPUs and thus obtaining additional Speedup for no cost”).

\section*{Abstrakt}
%TODO 
Táto práca analyzuje umelé neurónové siete (UNS), ktoré sú založené na Generalized recirculation algorithm (GeneRec)~\citep{o1996bio} a Bidirectional Activation-based Learning algorithm (BAL)~\citep{farkas2013bal}. Od štandardných sietí, akými sú napríklad siete spätne šíriace chybu (BP), sa líšia tým, že zmena váh je založená na rozdiely dopredných a spätných aktivácií. Takéto siete sa považujú za prirodzené pre ich obojsmernosť a preto, lebo šíria iba aktiváciu a nie chybu. Je známe, že tieto siete majú problémy s naučením aj jednoduchých úloh, ktoré sa BP vie naučiť. Cieľom práce je preto zvýšenie úspešnosti BALu. 

Analyzujeme viacero modifikácií BALu. Na základe pozorovaní navrhujeme model Two learning rates (TLR), ktorý využíva rozdielne rýchlosti učenia pre rôzne matice. Pomocou simulácií dokážeme, že TLR zvyšuje úspešnosť BALu. Navyše pozorujeme jasné závislosti medzi rýchlosťami učenia a úspešnosťou siete. Zaujímavosťou je, že pre najlepšie siete môže byť rozdiel medzi dvoma rýchlosťami učenia až $10^6$. Myšlienku TLR aplikujeme aj na GeneRec. Navyše skúšame viacero štandardných modifikácií UNS, ako sú napríklad moment, dávkové učenie, dynamická rýchlosť učenia alebo inicializácia váh. 

Veríme, že aplikácia myšlienky TLR má potenciál zvýšiť úspešnosť aj iných modelov UNS. Myšlienka sa dá zovšeobecniť aj na iné parametre, ako sú napríklad moment alebo inicializácia váh. 

\begin{flushleft}
  {\bf Kľúčové slová}: učenie s učiteľom, neurónová sieť, heteroasociatívne zobrazenie, dynamická rýchlosť učenia, učenie na základe aktivácií
\end{flushleft}

%keywords={ Internet; TCP streams; Tor network;}


\newpage
%\thispagestyle{empty}

    %Abstract is very important part of the thesis. It will be most read by people and should be written with a great care. The abstract should mention:
% 1) About the problem you want to solve
% 2) About your solution – how you solve the problem
% 3) Highlights about how good is your solution (e.g. achieves 70\% better performance) referring to the results you obtained in your experiments (e.g. achieves 70\% better performance).
% 4) Possible impacts of your work into the field (e.g. “The proposed solution can be used to offload the CPU by executing data parallel computation intensive code on GPUs and thus obtaining additional Speedup for no cost”).

\section*{Abstract}

% 1) About the problem you want to solve
We analysed artificial neural networks without error propagation based on the Generalized recirculation algorithm (GeneRec) by \citet{o1996bio} and the Bidirectional Activation-based Learning algorithm (BAL) by \citet{farkas2013bal}. The main idea of both algorithms is to update weights based on the differences between forward and backward activations. Thus both being more biologically plausible than standard models such as the Backpropagation algorithm (BP). On the other hand both algorithms have a considerable performance gap in comparison to BP.

% 2) About your solution – how you solve the problem
We introduced the Two learning rate model (TLR) which is based on BAL and uses different learning rates for different weight matrices. Our simulations proved increase in success rate from $\approx 65\%$ to $\approx 95\%$ while having a smooth relation between success and the learning rates. We achieved TODO\% on the well--known hand--written digits dataset. The most shocking being the fact that the two learning rates differed by magnitude of $10^8$ what we found hard to explain. We further applied the idea of TLR to GeneRec and BIA and experimented with momentum, weight initialization, hidden activations, dynamic learning rate, dropout and other. 

% 4) Possible impacts of your work into the field (e.g. “The proposed solution can be used to of
We believe that using the idea of TLR could lead to performance increase in other artificial neural network models and even several layered networks. The idea of different parameters for different weight matrices could be used also for momentum or weight initialization. We outline further experiments which should be done. 

\begin{flushleft}
  \textbf{Keywords:} supervised learning, artificial neural network, classification, heteroassociative mapping, dynamic learning rate, activation based learning, generec, bal, hand--written digits. 
\end{flushleft}

%keywords={ Internet; TCP streams; Tor network;}


    
% ==================== 5. Table of Contents ========================
\newpage
\setcounter{page}{1}
\pagenumbering{arabic}
\pagestyle{fancy}

\tableofcontents

% ==================== 6. Table of Figures ========================
%    \listoffigures
% ==================== 7. Table of Tables ========================
%    \listoftables
% ==================== 8. Table of Abbrevations ========================
% TODO 
%    \section*{Dictionary}
\markboth{DICTIONARY}{}    
\addcontentsline{toc}{section}{Dictionary}

\begin{itemize}
\item Differential equations - TODO learn the basics (to have an intuition for computing the learning rules). Continuous? Ask Ondrac for materials. 
\item Difference equations - TODO learn the basics (to have an intuition for computing the learning rules). Discrete? 
\item Antiparallel vectors - (Wiki, TODO) In a vector space over $\mathbb{R}$ (or some other ordered field), two nonzero vectors are called antiparallel if they are parallel but have opposite directions. In that case, one is a negative scalar times the other.
\item Kronecker delta - (Wiki, TODO) In mathematics, the Kronecker delta or Kronecker's delta, named after Leopold Kronecker, is a function of two variables, usually integers. The function is 1 if the variables are equal, and 0 otherwise: 
$$
    \delta_{ij} = \left\{\begin{matrix} 0, & \mbox{if } i \ne j \\ 1, & \mbox{if } i=j, \end{matrix}\right. $$

\item Steady state = fixed point. (Wiki, TODO) An important goal is to describe the fixed points, or steady states of a given dynamical system; these are values of the variable which won't change over time. Some of these fixed points are attractive, meaning that if the system starts out in a nearby state, it will converge towards the fixed point.

\item Periodical points. (Wiki, TODO) Similarly, one is interested in periodic points, states of the system which repeat themselves after several timesteps. Periodic points can also be attractive. Sharkovskii's theorem is an interesting statement about the number of periodic points of a one-dimensional discrete dynamical system.

\item Final mean root square weight per connection (Pineda, page 3). 

\item PCA - Principial component analysis. TODO. \cite{hinton1988learning} 

\item Auto encoder. TODO. \cite{hinton1988learning} 

\item Mean field annealing. Mean field annealing ( Soukoulis et al., 1983; Bilbro et al., 1989) is a deterministic approximation to simulated annealing which is significantly more computationally efficient (faster) than simulated annealing ( Bilbro et al., 1992). Instead of directly simulating the stochastic transitions in simulated annealing, the mean (or average) behavior of these transitions is used to characterize a given stochastic system. Because computations using the mean transitions attain equilibrium faster than those using the corresponding stochastic transitions, mean field annealing relaxes to a solution at each temperature much faster than does stochastic simulated annealing. \url{http://neuron.eng.wayne.edu/tarek/MITbook/chap8/8_4.html}

\end{itemize}

 
%    \makenomenclature http://tex.stackexchange.com/questions/100354/list-of-abbreviations
    
% ==================== 9. Introduction ========================
% This section should contain a little about everything. Introduction should be an overview of the contents of your thesis. 
\newpage
    %This section should contain a little about everything. Introduction should be an overview of the contents of your thesis. The introduction should contain:

% 1) Information/introduction about the topic of your research (e.g. what you will talk about in your thesis – Compilers, CPUs, optimizations etc.)

% 2) The practical and theoretical value of the topic (how and why this topic is important)

% 3) The motivation for your thesis (State with at least one sentence the problem you attack in your research work, why did you choose this problem and how it is interesting.  State with at least one sentence your solution for the problem).

% 4) If you are basing your work on currently existing work, mention it here.

% 5) Mention the limitations of your solution (design and implementation – e.g. applies for real time systems, has error factor 25%)

% 6) Include information about your key results – e.g. we improve the performance with 70% in the general keys.

% 7) Finish the chapter with an overview about the contents of your thesis.

\section*{Introduction}
\markboth{INTRODUCTION}{}    
\addcontentsline{toc}{section}{Introduction}

%Neural networks are a general method in Machine Learning used when everything other fails. 

UVOD - pre citatela, ktory temu trocha chape, zavedenie zakladnej notacie 
\begin{itemize} 
\item   co je problem
\item   preco zaujimave,
\item   co je zname,
\item   my sme urobili toto
\end{itemize} 

We analyse and improve BAL \ref{models-bal} in terms of success and epoch while being inspired by Generec \ref{models-generec}, CHL \ref{models-chl} and others (TODO). 


    
% ==================== 10. Motivation (or Problem Definition and Proposed Solution) =====
% In this chapter you have to concisely explain the problem that you want to solve and the goal of your solution. 
\newpage
    %In this chapter you have to concisely explain the problem that you want to solve and the goal of your solution. This part should contain:

% 1) Detailed analysis of the problem and its limitations (e.g. what is the bottleneck and difficulties).

% 2) Your research methods – how did you identify these problems (e.g. tools used)?

% 3) You should clearly state and explain your goal and objectives. You should provide analytical study (mathematical model) of your solution (What is the upper and lower bound of your performance or improvements). You should also mention about the qualitative benefit of your solution such as easy programming etc..

% If necessary you may divide this chapter in sections and subsections.

\subsection*{Motivation}
TODO add bidirectional (it could be that neurons in brain are connected that way) 

\subsubsection{O'Reilly} 
It corresponds to O'Reillys motivations \citet{o1998six}.

O'Reilly presents what he thinks as biologically plausible. In the end of this review we provide citations from this article which shortly explain the most important concepts of NN design. 

Article also contains interesting references to several experiments. It also presents the Leabra model (PhD thesis of O'Reilly) which is presented as a base model for other NN which can be derived from Leabra. The question how to merge the proposed principles is dicussed, especially the case of competiveness and distributed representation. 

\paragraph{Biological realism.} Moreover, computational mechanisms that violate
known biological properties should not be relied upon. 

A criticism of back-propagation is that it is neurally implausible (and hard to implement in hardware) because it requires all the connections to be used backward and it requires the units to use different input-output functions for the forward and backward passes \citet{hinton1988learning}.

\paragraph{Distributed representations.} A distributed representation
uses multiple active neuron-like processing units to encode
information (as opposed to a single unit, localist represen-
tation), and the same unit can participate in multiple repre-
sentations. Each unit in a distributed representation can be
thought of as representing a single feature, with information
being encoded by particular combinations of such features \citet{hinton1988learning}.

\paragraph{Inhibitory competition.} Inhibitory competition arises when mutual
inhibition among a set of units (i.e. as mediated by in-
hibitory interneurons) prevents all but a subset of them
from becoming active at a time.  Furthermore, most learn-
ing mechanisms (including those discussed later) are
affected by this selection process such that only the selected
representations are refined over time through learning, re-
sulting in an effective differentiation and distribution of
representations. More generally, it seems as though the world can be usefully
represented in terms of a large number of categories with a
large number of exemplars per category (animals, furniture,
trees, etc.) \citet{hinton1988learning}. 

\paragraph{Bidirectional activation propagation (interactivity).} They showed that
interactivity could explain the counterintuitive finding that
higher-level word processing can influence lower-level letter
perception. More recently, Vecera and O’Reilly showed
that bidirectional constraint satisfaction can model people’s
ability to resolve ambiguous visual inputs in favor of familiar
versus novel objects \citet{hinton1988learning}. 

\paragraph{Error-driven task learning.} Error-driven learning (also called ‘supervised’ learning) is
important for shaping representations according to task de-
mands by learning to minimize the difference (i.e. the error)
between a desired outcome and what the network actually
produced \citet{hinton1988learning}. 

\paragraph{Hebbian model learning.} That something like correlational structure is important.
Hebbian learning mechanisms represent this correlational
structure, encoding the extent to which different things co-
occur in the environment \citet{hinton1988learning}.

\subsubsection{Da} 

\citet{da2011advances} 

There is evidence that the cerebral cortex is connected in a
bi-directional way and distributed representations prevail in
it \citet{o2000computational}. So, more biologically plausible connectionist models
should present some of the following characteristics \citet{orru2008sabio}:

\paragraph{Distributed representation.} Generalization and reduction
of the network size can be obtained if the adopted
representation is distributed, since connections among
units are able to support a large number of different
patterns and create new concepts without allocation of
new hardware;

\paragraph{Inhibitory competition.} Neurons that are next to the
“winner” neuron receive a negative stimulus, strengthen-
ing the winner neuron (a kind of winner-takes-all). Dur-
ing a lateral inhibition, a neuron excitets an inhibitory
inter-neuron that makes a feed-back connection on the
first neuron \citet{o1998six};

\paragraph{Bi-directional activation propagation.} Hidden layers re-
ceive stimuli from the input layer and from the output
layer. The bi-directionality of the architecture is neces-
sary to simulate a biological electrical synapse, which
can be bi-directional \citet{kandel1995essentials}, \citet{rosa2002biologically};

\paragraph{Error-driven task learning.} In GeneRec, the error is
calculated considering the local difference in synapses,
based on neurophysiological properties \citet{o1998six}, in contrast
to back-propagation, which requires error signals prop-
agate from the output layer towards the input layer.

\subsubsection{Usage} 

BioAnt \citet{schneider2009application} 
Cells \citet{nawrocki2012monitoring} 


\subsubsection{Other} 
\paragraph{Hebbian nature.}
%TODO make it a lot shorter 
%TODO: Read and citet from Hebbs's original article (Hebb, D.O. (1949). The organization of behavior. New York: Wiley \& Sons). 

(Wiki) Hebbian theory is a scientific theory in biological neuroscience which explains the adaptation of neurons in the brain during the learning process. It describes a basic mechanism for synaptic plasticity wherein an increase in synaptic efficacy arises from the presynaptic cell's repeated and persistent stimulation of the postsynaptic cell. Introduced by Donald Hebb in 1949, it is also called Hebb's rule, Hebb's postulate, and cell assembly theory, and states:

    "Let us assume that the persistence or repetition of a reverberatory activity (or "trace") tends to induce lasting cellular changes that add to its stability. When an axon of cell A is near enough to excitet a cell B and repeatedly or persistently takes part in firing it, some growth process or metabolic change takes place in one or both cells such that A's efficiency, as one of the cells firing B, is increased."

The theory is often summarized as "Cells that fire together, wire together.".[1] It attempts to explain "associative learning", in which simultaneous activation of cells leads to pronounced increases in synaptic strength between those cells. Such learning is known as Hebbian learning.

\includegraphics[width=12cm]{img/bio_plausability_o1998six.png}

    
% ==================== 11. State of the Art (or Related Work or Literature Review) =========
% In this section you should present the theoretical basis of your work and overview of the existing solutions. When discussing on the exiting solutions you should relate and qualitatively compare them with yours.
\newpage
    %11.   State of the Art (or Related Work or Literature Review)
%In this section, you should present the theoretical basis of your work and overview of the existing solutions. When discussing on the exiting solutions you should relate and qualitatively compare them with yours. This chapter should contain:

% 1) The theory and concepts of your work. For example, if you work on compiler you can mention about how compiler works without getting much into detail.

% 2) Existing state of the art solution. For example, if you implement an optimization pass, you should mention about existing optimization passes that are relevant with yours. You should emphasize the commons and differences with your solution and with the existing ones.

% You may divide this chapter in sections (e.g. each for the different existing solution).

\section{Overview}
\label{sec:overview} 

\subsection{Preliminaries}
\label{sec:theory} 

% The theory and concepts of your work. For example, if you work on compiler you can mention about how compiler works without getting much into detail.

\newcommand{\argmin}{\operatornamewithlimits{arg\,min}}
\newcommand{\Bx}{{\bf x} }
\newcommand{\By}{{\bf y} }
\newcommand{\Bh}{{\bf h} }
\newcommand{\Bw}{{\bf w} }
\newcommand{\Bc}{{\bf c} }

In this section,we will describe the basics of artificial neural networks. We will also introduce the notation used in this work. Note that the definitions and notations vary through the literature. We use the one which the author is familiar with. For the reader who is comfortable with this topic we recommend to skip this section and go to related models~(\ref{sec:overview-models}). 

%=============================================================
\subsubsection{Perceptron.}

The theory behind artificial neural networks started with the model of \emph{Perceptron} introduced by \citet{mcculloch1943logical}. It is a simple model which transforms a vector of inputs $s$ to an output value $y$. 

\begin{figure}[h]
  \centering
  \includegraphics[width=0.7\textwidth]{img/perceptron.pdf}    
  \caption{Perceptron. Notation: $x$ is the \emph{input vector} where always $x_0=1$, $w_{0k}$ is the \emph{weight} vector, $\Sigma$ is the \emph{summing} junction, $\eta_k$ is the \emph{net input}, $\phi$ is the \emph{activation function}, $\theta_k$ is the \emph{treshold}, $y_k$ is the \emph{output} and $b_k$ is the \emph{bias}.} 
  \label{fig:perceptron}
\end{figure}

The whole transformation of the input vector to the output activation could be written as follows: 
\begin{equation}
\label{eq:perceptron} 
y_k =
\left\{
	\begin{array}{ll}
		0 & \mbox{if } \phi(\sum_{i=0}^N x_iw_{ik}) < \theta_k \\
		1 & \mbox{otherwise}
	\end{array}
\right.
\end{equation} 

Equation~\ref{eq:perceptron} describes a simple \emph{binary treshold perceptron}. One could observe that the binary perceptron divides the vector space $\mathbb{R}^N$ by a $(n-1)$--dimensional hyperplane. This behaviour was studied by \citet{rosenblatt1958perceptron}. Now we see the importance of bias which is the absolute term in the equation of the hyperplane. 

\paragraph{Continuous perceptron.}
We put additional constraints for the activation function $\phi \mathbb{R} \mapsto (0,1)$: 
\begin{enumerate} 
\item It is differentiable and monotonously increasing,
\item and satisfying two asymptotic conditions $t(-\infty)=0$ and $t(\infty)=1$. 
\end{enumerate} 
Usually, the transfer function is realized by the logistic function: 

\begin{equation}
\frac{1}{1 + e^{-\eta}}. 
\end{equation} 

To allow the outputs to be from range $(0,1)$ we drop the treshold and simply output $\phi(\eta_k)$. 

\paragraph{Learning.} 
The goal of a perceptron is to \emph{learn} a given set $T = \{(X^j, t_j)\}$ mappings, where $X^j$ is the input vector $(x_{j0},x_{j1}, \ldots, x_jN)$ and $t_j$ is the corresponding target. Ideally, we want from the perceptron to be able generalize for novel inputs. The goal is formalized by minimizing an error function: 

\begin{equation}
\label{eq:perceptron-error} 
E = \sum_{k=1}^{N} \frac{1}{2}(y_k-t_k)^2.
\end{equation} 

A straightforward method to achieve this is simply updateing weights according to the partial derivates of the error function: 

\begin{equation}
\label{eq:perceptron-learning} 
\frac{\partial E}{\partial w_{ik}} = (y_k - t_k)\phi'(\eta_k)x_i = (y_k - t_k)y_k(1 - y_k)x_i.
\end{equation} 

From equation~\ref{eq:perceptron-learning} we can develop a following learning algorithm: 
\begin{figure}[h]
  \centering
\begin{lstlisting}
for e from 1 to EPOCHS: 
  foreach (X^j, t_j) in T: 
    calculate y_j
    for i from 0 to N: 
      update w_{ij}
\end{lstlisting}
  \caption{Perceptron learning. Developed from equations~\ref{eq:perceptron-error} and \ref{eq:perceptron-learning} we get an \emph{weight update} rule which is applied in loop for element in $T$. TODO sth better than lstlisting} 
  \label{fig:perceptron-learning}
\end{figure}

Usually, the \emph{update rule} is written as: 
\begin{equation} 
\Delta w_{ik} = \lambda (y_k - t_k)y_k(1 - y_k)x_i,
\end{equation} 
where $\lambda$ is the \emph{learning speed}. 

 
\label{sec:perceptron} 

\subsubsection{Multilayer Feedworward Networks} 
\label{sec:theory-multilayer} 

We will define \emph{multilayer feedforward networks} as in~\citet{haykin1994neural}. First, we define a \emph{layered} neural network where neurons are organised to form layers. In the simplest version we have an \emph{input layer} of source nodes and an \emph{output layer} which is formed by aforementioned perceptrons. In other words this is a \emph{feedforward} or \emph{acyclic} type of network as the \emph{activation}, i.e. outputs of the neurons are computed from the input to the output layer and never \emph{backwards}. 

\begin{figure}[H]
  \centering
  \includegraphics[width=0.5\textwidth]{img/multilayer.pdf}    
  \caption{Fully connected feedforward \emph{multilayer} network with one \emph{hidden} layer. } 
  \label{fig:multilayer}
\end{figure}

Multilayer neural network has one or more \emph{hidden layers} in addition to the input and ouput layer as shown on figure~\ref{fig:multilayer}. The source nodes supply the activation pattern, i.e. input vector, which is applied to next layer of neurons. The output signal of the hidden layer is used as the input for the output layer. As shown by~\citet{cybenko1989approximation} the three layer network is an universal approximator of continuous functions on compact subsets of $\mathbb{R}^n$.

There exists several methods for training multilayer networks. First, we will describe the most common backpropagation in~(\ref{sec:models-bp}) and then methods related to our work such as CHL~(\ref{sec:models-chl}), GeneRec~(\ref{sec:models-generec}) and BAL~(\ref{sec:models-bal}). 


\subsubsection{Recurrent networks}
\label{sec:theory-recurrent} 

In \emph{recurrent} neural networks also cycles of connections are allowed. This arises problems with computing their activations. That means that output of a particular unit could affect its input. Therefore, the activations in general could not be computed only by one forward pass. This introduces real valued dynamic systems for computing the activations. We can observe that it holds that $\partial\eta / \partial t = 0$ for the activations of neurons in the fixed point state. There are several approaches solving these dynamic systems and deriving the learning rule~\citep{pineda1987generalization, pearlmutter1989learning, williams1989learning, elman1990finding, haykin1994neural}. 

\begin{figure}[H]
  \centering
  \includegraphics[width=0.4\textwidth]{img/models-recurrent.pdf}    
  \caption{Simple recurrent network proposed by~\citet{elman1990finding}. Taken from~\citet{haykin1994neural}.} 
  \label{fig:theory-recurrent}
\end{figure}

An \emph{iterative method} is used by~\citet{movellan1990contrastive} for computing activations. In the first step the input neurons have activations equal to the input vector and the other neurons have zero activation. In the next steps activations from the last step are used to compute activation in this step as shown in equation~(\ref{eq:theory-recurrent-activation}): 
\begin{equation}
  \label{eq:theory-recurrent-activation} 
  \eta_i(t+1) = \phi\left(\sum_j w_{ji}\eta_i(t)\right) 
\end{equation}
This rule is iterated while the activations are not settled. For particullar symmetric networks it could be proved that activations will converge~\citep{o1996bio}. For more general networks a dynamic system based on rule~(\ref{eq:theory-recurrent-activation}) could be introduced. The the fixed point solution is the settled activation. \citet{movellan1990contrastive} proposes using the method of simulated annealing~\citep{kirkpatrick1983optimization,vcerny1985thermodynamical} to improve the learning rule and to avoid settling the network in a local minima. We experimented with the iterative method for a two way version of GeneRec in~Section~(\ref{sec:our-bal-recirc}). 

 

\subsubsection{Hopfield networks}
\label{sec:theory-hopfield}

\citet{hopfield1984neurons} introduced a network with arbitrary connections defined only by one weight matrix $W$. Some of the units are chosen as the \emph{input units} which have stable activations for a given input pattern. We can treat a Hopfield network as a recurrent neural network. A Hopfield network comes with a continuous energy function for which usually function~(\ref{eq:theory-hopfield-energy}) is chosen: 
\begin{equation}
  \label{eq:theory-hopfield-energy}
  E = -\frac{1}{2}\sum_i\sum_ja_iw_{ij}a_j,
\end{equation} 
where $a_i$ is the activation of the $i$-th unit. The aim of the network is to settle the activations so that $E$ settles in a global minima. Activation for the $i$-th unit is computed based on the following differential equation~\citep{hopfield1984neurons}: 
\begin{equation}
  \label{eq:theory-hopfield-activation}
  \frac{\partial a_i}{\partial t} = \alpha(-a_i + f_i(\eta_i)),
\end{equation} 
where $a^T = [a_1,\ldots,a_n]$ is the activation vector, $f_i$ is bounded, monotically increasing, differentiable activation function. \citet{hopfield1984neurons} proved for equation~(\ref{eq:theory-hopfield-activation}) that if the weights are symmetric, i.e. $w_{ij} = w_{ji}$, the activations will settle in the minimal error state defined in equation~(\ref{eq:theory-hopfield-activation}). This learning rule is typically used in \emph{interactive activation networks} studied by~\citet{grossberg1978theory} and~\citet{mcclelland1981interactive}. 

 
 

\subsubsection{Backpropagation}
\label{models-bp} 

TODO spomenut hetero-asociativne, jednosmerne, supervised 
TODO napisat ako theory example

A criticism of backpropagation is that it is neurally implausible (and hard to implement in hardware) because it requires all the connections to be used backward and it requires the units to use different input-output functions for the forward and backward passes \citet{hinton1988learning}.

The procedure repeatedly adjusts the weights of the connections in the network so as to minimize a measure of the difference between the actual output vector of the net and desired output vector \citet{rumelhart1986learning}. 

The aim is to find a powerful synaptic modification rule that will allow an arbitrarily connected neural network to develop an internal structure that is appropriate for a particular task domain \citet{rumelhart1986learning}. 

Connection within a layer or from higher to lower layers are forbidden, but connections can skip intermediate layers \citet{rumelhart1986learning}.

All units within a layer have their states set in parallel, but different layers have their states set sequentially, starting at the bottom and working upwards until the states of the output units are determined \citet{rumelhart1986learning}. 

$$x_j = \sum_i y_iw_{ji}.$$

$$y_i = \frac{1}{1 + e^{-x_i}}.$$

$$E = \frac{1}{2} \sum_c \sum_j (y_{j,c} - d_{j,c})^2,$$
where $c$ is index over cases. 

To minimize $E$ by gradient descent it is necessary to compute the partial derivate of $E$ with respect to each weight in the network \citet{rumelhart1986learning}. 

$$\partial E / \partial y_j = y_j - d_j.$$

We computed $\partial E / \partial y_j$ for any unit when $\partial E / \partial y_j$ given in the last layer. Repeating this procedure we get $\partial E / \partial y_j$ for all weights. 

Adding momentum:
$\Delta w(t) = -\epsilon \partial E/ \partial w(t) + \alpha \Delta w(t-1).$

Adding a few more connection creates extra dimensions in weight-space and these dimensions provide paths around the barriers that create poor local minima in the lower dimensional subspaces \citet{rumelhart1986learning}. 

The learning procedure, in its current form, is not a plausible model of learning in brains. 

$$\frac{\partial E}{\partial w_{ij}} = -\sum_k(t_k-o_k)w_{jk}\sigma'(\eta_j)s_i,$$
where $t_k$ is the target value, $o_k$ is the output value, $\sigma$ is the nonlinear function, $\eta_j$ is the net input and $s_i$ is the stimulus input \citet{o1996bio}.

%TODO prekreslit do IPE 
\begin{center} 
\includegraphics{img/table_bp.png} 
\citet{farkas2013bal} 
\end{center} 


\subsection{Related models}
\label{sec:overview-models}  

In this section, we briefly mention models used in our work. Mainly it~is the Bidirectional Activation-based Learning algorithm~(\ref{sec:models-bal}) by~\citet{farkas2013bal} and the Generalized recirculation~(\ref{sec:models-generec}) by~\citet{o1996bio}. The other two models are inspiration for the two former ones. Understanding the latter helps understanding the former. We compare these models to our specialized versions in Table~\ref{tab:results-cmp-auto4}. 

%\subsubsection{Boltzmann machines}
TODO \cite{ackley1985learning}

TODO: Read and cite from Hinton's original article. 

(Wiki) A Boltzmann machine is a type of stochastic recurrent neural network invented by Geoffrey Hinton and Terry Sejnowski. Boltzmann machines can be seen as the stochastic, generative counterpart of Hopfield nets. They were one of the first examples of a neural network capable of learning internal representations, and are able to represent and (given sufficient time) solve difficult combinatoric problems. If the connectivity is constrained, the learning can be made efficient enough to be useful for practical problems.

A Boltzmann machine, like a Hopfield network, is a network of units with an "energy" defined for the network. It also has binary units, but unlike Hopfield nets, Boltzmann machine units are stochastic. The global energy, $E$, in a Boltzmann machine is identical in form to that of a Hopfield network:

$$E = -\sum_{i<j} w_{ij} \, s_i \, s_j - \sum_i \theta_i \, s_i.$$

Where:
\begin{itemize}
    \item $w_{ij}$ is the connection strength between unit $j$ and unit $i$.
    \item $s_i$ is the state, $s_i \in \{0,1\}$, of unit $i$
    \item $\theta_i$ is the threshold of unit $i$.
\end{itemize}

The connections in a Boltzmann machine have two restrictions:
\begin{itemize}
    \item $w_{ii}=0\qquad \forall i$. (No unit has a connection with itself.)
    \item $w_{ij}=w_{ji}\qquad \forall i,j$. (All connections are symmetric.)
\end{itemize}

Often the weights are represented in matrix form with a symmetric matrix $W$, with zeros along the diagonal.

\paragraph{Hopfield nets.}

TODO: Read and cite from Hopfields's original article. (Storkey, Amos. "Increasing the capacity of a Hopfield network without sacrificing functionality." Artificial Neural Networks—ICANN'97 (1997): 451-456.)

(Wiki) A Hopfield network is a form of recurrent artificial neural network invented by John Hopfield. Hopfield nets serve as content-addressable memory systems with binary threshold nodes. They are guaranteed to converge to a local minimum, but convergence to a false pattern (wrong local minimum) rather than the stored pattern (expected local minimum) can occur. Hopfield networks also provide a model for understanding human memory.

\paragraph{Boltzmann distribution.}

TODO:  Landau, Lev Davidovich; and Lifshitz, Evgeny Mikhailovich (1980) [1976]. Statistical Physics. 5 (3 ed.). Oxford: Pergamon Press. ISBN 0-7506-3372-7. Translated by J.B. Sykes and M.J. Kearsley. See section 28

(Wiki) The Boltzmann distribution for the fractional number of particles $Ni / N$ occupying a set of states $i$ possessing energy $E_i$ is:

    $${N_i \over N} = {g_i e^{-E_i/(k_BT)} \over Z(T)}.$$

where $k_B$ is the Boltzmann constant, $T$ is temperature (assumed to be a well-defined quantity), $g_i$ is the degeneracy (meaning, the number of levels having energy $E_i$; sometimes, the more general \'states\' are used instead of levels, to avoid using degeneracy in the equation), $N$ is the total number of particles and $Z(T)$ is the partition function.

    $$N=\sum_i N_i,$$

    $$Z(T)=\sum_i g_i e^{-E_i/(k_BT)}. $$
  %moved to old 

%TODO equilibrium values 
\def\myover#1#2{\mathrel{\overset{\makebox[0pt]{#2}}{#1}}}
\newcommand{\mytilde}{\raise.17ex\hbox{$\scriptstyle\mathtt{\sim}$}}
\def\myequi#1{\myover{#1}{\mytilde}}

\subsubsection{Contrastive Hebbian Learning}
\label{sec:models-chl} 

The main idea of \emph{Contrastive Hebbian Learning} developed by \citet{movellan1990contrastive} is to have two activation phases in an aribtrary Hopfield network \citep{hopfield1984neurons} described in \ref{sec:theory-hopfield}. In the first phase, called \emph{minus phase} and denoted \quotes{-}, only the input vector is \emph{clamped}, i.e. activations of the clamped units as equal to the clamped values. In the second phase, called \emph{plus phase} and denoted \quotes{+}, both the input and target are clambed to the underlying network. The learning is based on the difference of these two activations. Note that CHL brings no assumptions about the structure of the underlying network and therefore it has no layers in general. 

As mentioned previously CHL is based on Hopfield networks. Therefore it has an energy function $J$ which is based on the Helmholtz free energy function $F$ \citep{hinton1989deterministic}:
\begin{equation}
  \label{eq:models-chl-helmholtz}
  F = -\frac{1}{2}\sum_i\sum_ja_iw_{ij}a_j + \sum_i \int_{f(0)}^{a_i} f_i^{-1}(a)da
\end{equation} 
where $-\frac{1}{2}\sum_i\sum_ja_iw_{ij}a_j$ is the Hopfield energy function~\ref{eq:theory-hopfield-energy}. The \emph{contrastive} error function $J$ is defined as: 
\begin{equation}
  \label{eq:models-chl-energy}
  J = \myequi{F^{+}} - \myequi{F^{-}}
\end{equation} 
where $\myequi{F^{+}}$ and $\myequi{F^{-}}$ respectively are the values of the energy functions at equilibrium states for the plus and the minus phases. 

Based on the contrastive energy function~\ref{eq:models-chl-energy} a learning rule is derived by \citet{movellan1990contrastive}: 
\begin{equation}
  \label{eq:models-chl-learning-rule}
  \Delta w_{ij} = \myequi{a_i^{+}}\myequi{a_j^{+}} - \myequi{a_i^{-}}\myequi{a_j^{-}}
\end{equation}
where $\myequi{a_i}$ and $\myequi{a_j}$ denote the equilibrium state activations of the $i$--th and $j$--th unit. It could be shown that the learning rule~\ref{eq:models-chl-learning-rule} decreases the energy function \ref{eq:models-chl-energy} \citep{movellan1990contrastive}. Moreover it could be shown that the CHL learning rule is equivalent to Backpropagation learning rule in terms of computability while it is biologically more plausible \citep{o1996bio, xie2003equivalence}. 

   


%\subsubsection{Almeida-Pineda Algorithm}

Pineda first applied the backpropagation rule to recurrent networks. Recurrent networks can have cycles and they contain no layers. Subset $\Omega$ of the units is treated as the output layer. So the Almeida-Pineda network is a generealization of the backpropagation network. 

As the network is recurrent it can remember information and can be used as CAM.
%TODO citation for CAM 

\includegraphics[width=8px]{img/recurrent.png}

\paragraph{My notes.} 
So the Pinedas NS is a dynamical system which should converge and converges for forward/backward nets (quite trivially). 

The delta rule is constructed to move the state towards the fixed point which is defined so that the difference on output neurons is equal to zero (values on other neurons are chosen arbitrary?). 

First Pineda derives an exact learning rule which requires computation of inverse matrix what is both bad for implementation and biological plausibility. So it is simplified to associate dynamical system (I don't understand how) which converges if the original dynamical system converges (proved by Almeida). 

It is more natural as all neurons are equivalent by construction. The advantage is exploited in hardware computation as it treats differential equations more naturally can be solved by analog computers. 

%TODO: Compare this conclusion with O'Reillys. 


TODO: Understand the basics of differential equations to enhance the intuition about learning rules. 

TODO: Study the Lapedes and Farber: master / slave NS. 

\paragraph{Citations from the article.}

Nevertheless it has been applied to recurrent networks by taking advanage of the fact that for every recurrent network there exists an equivalent feedforward network (for a finite time) \cite{pineda1987generalization}.

Hopfield's equations are globally asymptotically stable if $w$ is symmetric and has zeros along the diagonal \cite{pineda1987generalization}.

$$y_r = \beta f_r^,(u_r)\sum_k J_k(L^{-1})_{kr},$$
where 
$$L_{ij} = \alpha \delta_{ij} - \beta f_i^,(u_i)w_{ij},$$
and where $\delta_{ij}$ is the Kronecker $\delta$ symbol and $J_i = t_t - x_i$ if $i \in \Omega$ and $J_i = 0$ otherwise \cite{pineda1987generalization}. 

Then the exact learning rule is 
$$dw_{rs}/dt = \gamma y_r x_s.$$
Rhis exact learning rule needs matrix inversion to calculate the error signals $y_k$. Direct matrix inversions are necessarily nonlocal calculations and therefore this learning algorithm is not suitable for implementation as a neural network \cite{pineda1987generalization}. 

TODO: Understand the associated dynamical system and the new learning rule (page 3/4). 

\paragraph{O'Reillys conclusion}
\cite{o1996bio}
The activation states in AP are updated according to a discrete-time approximation of the following dif-
ferential equation, which is integrated over time with respect to the net input terms :

$$\frac{d\eta_j}{d_t} = -\eta_j + \sum w_{ij} \sigma(\eta_i).$$

This equation can be iteratively applied until the network settles into a stable equilibrium state (i.e., until the
change in activation state goes below a small threshold value), which it will provably do if the weights are
symmetric (Hopfield, 1984), and often even if they are not (Galland \& Hinton, 1991).

\includegraphics[width=10cm]{img/table_ap.png}
  %moved to old 

\subsubsection{Recirculation algorithm}
\label{models-recirc} 

TODO image of the algorithm 
TODO shorten and rewrite the text 
TODO spomenut ze unsupervised 

We will outline the main ideas used for this Recirculation algorithm. They motivation is in an autoencoder (or data-compression) and they also mention the PCA algorithm. 

First, there are just two layers - visible and hidden which are connected by directed weights (most of time we assume these weights as symmetric). 

Second, we operate in discrete time. Each learning phase has four steps. In first two the input is propagated to hidden layer. In second two the desired output is propagated to hidden layer. The learning on visible layer occurs in second step when the desired output meets the reconstructed input. Similary, the learning on hidden layer occurs in step 3.

Third, the learning rules for both layers is almost identical (we can non-formally write that they are symmetric). It is interesting that the learning rule uses no exact ifnormation about the non-linear function (usually logistics), but only assumes that it has bounded derivate. The learning rule is an approximation of the learning rule derived from the difference equations for error correction. 

Authors study the case of asymetric weights. They do not provide a proof of convergence but they provide some intutition why it should work. Also they link the work of Ballard who experimented with connecting (merging) several closed loops so that hidden units of closed loops can be input units of other closed loops of recirculation.

\paragraph{From the original article.}
Instead of using a separate group of units for the input and output we use the very same group of \textit{visible} units, so the input vector is the initial state of this group and the output vector is the state after information has passed around the loop. The difference between the activity of a visible unit before and after sending activity around the loop is the derivative of the squared reconstruction error \citet{hinton1988learning}.

On the first pass, the original visible vector is passed around the loop, and on the second pass ana average of the original vector and the reconstructed vector is passed around the loop. The learning procedure changes each weight by an amount proportional to the product of the \textit{presynaptic} activity and the \textit{difference} in the post-synaptic activity on the two passes  \citet{hinton1988learning}.

$$\Delta w_{ij} = \epsilon y_j(1)[y_i(0)-y_i(2)],$$
$$\Delta w_{ji} = \epsilon y_i(2)[y_j(1)-y_j(3)],$$
where $y_i(t)$ is the activation value of unit $i$ in time $t$.

With approximation we can derive the following learning rule
$$\frac{\partial E}{\partial w_{ji}} \approx \frac{1}{1-\lambda}y_i(2)[y_j(3)-y_j(1)].$$
An interesting property of this equation is that it does not contain a term for the gradient\footnote{We assume that it is differentiable, it is monotonous and has bounded derivate.} of the input-output function of unit $j$ so recirculation learning can be applied even when unit $j$ uses an unknown non-linearity \citet{hinton1988learning}. 

%TODO: Understand why exactly we assume the linearity of the visible layer.

%TODO: Go through the derivation of the learning rule to understand why it is working as the gradient descent. 

\paragraph{O'Reillys conclusion.}

$$\frac{\partial E}{\partial h_j} = - (\sum_k t_k w_{jk} - \sum_k o_k w_{jk})$$
Thus, instead of having a separate error-backpropagation phase to communicate error signals, one can think in terms of standard activation propagation occuting via reciprocal (and symmetric) weights that come from the ouput units to the hidden units \citet{o1996bio}. 

\includegraphics[width=15px]{img/recirculation.png}


%\input{models-deep} %moved to old 

\subsubsection{GeneRec}
\label{models-generec} 

\paragraph{Introduction - version 1} 
The algorithm GeneRec (Generic Recirculation), developed by O’Reilly \citet{o1996bio} based on back-propagation, is argued to be a more biologically plausible supervised learning algorithm: learning happens through synaptic weight modifications using only local information available in synapses. In summary, GeneRec is a generalized version of the recirculation algorithm \citet{hinton1988learning}, which overcomes the limitations of the earlier algorithm (ex.: back-propagation) by using a generic recurrent network with sigmoidal units that can learn arbitrary input and output mappings. GeneRec employs two phases: \emph{minus} and \emph{plus} \citet{da2011advances}. 

\paragraph{Introduction - version 2} 
Learning is done by the Generalized Recirculation
(GeneRec) algorithm, which is argued to be a more bio-
logically plausible form of learning, developed by O’Reilly
[9]. One motivation of the development of GeneRec was
the treatment of typical problems encountered in the
backpropagation algorithm when working with bidirec-
tional networks. There is propagation of two signals in
GeneRec: the expectation of the network (called \emph{minus}
phase) and the training signal (also called \emph{plus} phase).
\citet{schneider2009application} 

\begin{table}
  \centering
  \caption{Equilibrium network variables in GeneRec model \citet{farkas2013bal}.}
  \label{tab:gr-states}
  \begin{tabular}{|cccc|}
    \hline
    Layer & Phase & Net Input & Activation\\
    \hline
    Input (s)    & $-$ & - & $s_i$ = stimulus input\\
    \hline
    Hidden (h)   & $-$ & \hspace{0.3cm}$\eta^{-}_j = \sum_i w_{ij}s_i + \sum_k w_{kj}o^{-}_k$\hspace{0.3cm} &
    $h^{-}_j = \sigma(\eta^{-}_j)$\hspace{0.3cm}\\
          &  +  & $\eta^{+}_j = \sum_{i}w_{ij}s_i + \sum_k w_{kj}o^{+}_k$ & $h^{+}_{j} = \sigma(\eta^{+}_j)$ \\
    \hline
    Output (o) & $-$ & $\eta^{-}_k = \sum_j w_{jk}h_j$ & $o^{-}_k = \sigma(\eta^{-}_k)$\\
           &  +  & - & $o^{+}_k$ = target output \\
    \hline
  \end{tabular}
\end{table}

The basic weight update rule in GeneRec is:
\begin{equation}
  \Delta w_{pq} = \lambda \ a^{-}_p(a^{+}_q - a^{-}_q)
\label{eq:generec}
\end{equation}
where $a^{-}_p$ denotes the presynaptic and $a^{-}_q$ denotes the postsynaptic unit activation in minus phase, $a^{+}_p$ is the presynaptic activation from plus phase (in output-to-hidden direction) and $\lambda$ denotes the learning rate. The learning rule given in Eq.~\ref{eq:generec} is applied to both input-hidden and hidden-output weights.  Due to the lack of space, the reader is left to consult the original paper \citet{o1996bio} regarding the underlying math behind the derivation of the GeneRec learning rule.

\paragraph{Minus phase.} When units xi are presented to the input layer A, there is the propagation of this stimulus to the hidden layer B (bottom-up propagation) (figure 1). At the same time, the previous output ok propagates from the output layer C to the hidden layer B (top- down propagation) (figure 2). Then the hidden activation \emph{minus} - minus phase - ($h^-$ ) is generated (sum of bottom-up and top-down propagations). The activation function $\sigma$ is sigmoid. Equation 1 shows the hidden activation calculus for one hidden unit j. wij are the synaptic weights from the input layer to the hidden layer and wjk are the synaptic weights from the hidden layer to the output layer (which are the same as the weights from the output layer to the hidden layer, because reciprocal weights are symmetric in GeneRec, that is, wjk = wkj \citet{o1996bio}). ok(t-1) is the previous output (output on time t - 1) \citet{orru2008sabio}.

$$h_j^- = \sigma \left(\sum_{i=0}^A w_{ij} \cdot x_i + \sum_{k=1}^C w_{jk} \cdot o_k(t-1)\right)$$

Finally, the real output ok(t) is generated through the propagation of the \emph{minus} layer activation to the output layer (figure 3), shown for one output unit k by equation 2 \citet{o1996bio}. Notice that the architecture employed is bi-directional. Recall that ok (t) (the current output on time t) is used in order to differentiate it from ok (t - 1) (the previous output on time t - 1).

\begin{center} 
\includegraphics{img/generec_minus_phase.png} \citet{orru2008sabio} 
\end{center} 

\paragraph{Plus phase.} Units xi are presented again to the input layer A; there is the propagation of this stimulus to the hidden layer B (bottom-up propagation) (figure 4). At the same time, the desired output yk propagates from the output layer C to the hidden layer B (top- down propagation) (figure 5). Then the hidden activation + \emph{plus} - plus phase - (hj) is generated, summin bottom-up and top-down propagations (equation 3) \citet{o1996bio}, \citet{orru2008sabio}.

$$h_j^+ = \sigma\left( \sum_{i=0}^A w_{ij} \cdot x_i + \sum_{k=1}^C w_{jk} y_k \right)$$

\begin{center} 
\includegraphics{img/generec_plus_phase.png} \citet{orru2008sabio} 
\end{center} 

In order to make learning possible, synaptic weights $w$ are updated, based on h-, j  $h^+_j$, $o_k$ , $y_k$, $x_i$, and the learning rate $\eta$ (equations 4 and 5).

%TODO equation labels with intext references 
$$\Delta w_{jk} = \eta(y_k - o_k(t)) h^-_j $$

$$\Delta w_{ij} = \eta(h^+_j - h^-_j) x_i$$

Finally, O’Reilly \citet{o1998six} suggests, unlike backpropagation, that the teaching signal is just another state of “experience” in the network, that is, in GeneRec algorithm the teaching signal is exactly the “top-down” activation in the plus-phase.


%TODO reformulate 
It was recently shown that backpropagation can be implemented in a more biologically plausible fashion using bidirectional activation propagation in an interactive network using the GeneRec algorithm \citet{o1996bio}, which is a generalization of the recirculation algorithm \citet{hinton1988learning}. In GeneRec, error information is propagated as two separate terms via standard activation propagation mechanisms in interactive networks, and the difference between these terms (which is the error signal) can be plausibly computed using the synaptic modification mechanisms underlying long term potentiation and depression (LTP/LTD). Versions of the GeneRec algorithm are equivalent to the other known ways of implementing powerful error-driven learning using interactive activation propagation instead of direct error propagation (e.g., the deterministic Boltzmann machine \citet{hinton1989deterministic} and Contrastive Hebbian Learning \citet{movellan1990contrastive}). Thus, several different approaches converge on the idea that the way to perform error-driven learning in a more biologically plausible manner is to use interactive networks, where error signals are communicated via top-down activation propagation \citet{o2001generalization}.

Copy from \citet{da2011advances} 


%TODO try out the following rules: 
Hebbian learning is performed using a Conditional Principal Components Analysis (CPCA) algorithm with a correction factor for sparse expected activity levels [3]. The error-driven learning is achieved with GeneRec; the output is computed in two phases – an expectation phase where the network's actual output is produced and an outcome phase where the target output is experienced – as a difference of a pre- and postsynaptic activation product across these two phases. Hebbian weights are adjusted according to the following formula.

equation image	(2)
while error-driven learning uses the following equation.

equation image	(3)
where xi is the input of neuron i, yj is the output of neuron j, and wij is the connection weight between neurons i and j. The “+” and “– superscripts refer to plus and minus phases of the GeneRec algorithm \citet{nawrocki2012monitoring}.

Some improvements \citet{da2008biological}. 

To assess the effect of interactivity, two different networks were compared on the combinatorial generalization task, a standard feedforward backpropagation network, and an interactive GeneRec network using the symmetric, midpoint variation learning rule which is equivalent to contrastive Hebbian learning (CHL) or a deterministic Boltzmann machine (DBM) \citet{o1996bio}, \citet{o2001generalization}. 
 
\footnote{TODO: How to cope with biases (which are not symmetric)? We haven't found how GeneRec uses the Bias neuron.} 


%==================== 12.   Background overview (optional) ======
\subsection{Bidirectional Activation-based Learning algorithm} 
\label{models-bal} 
% If your work builds on top of an existing one, this is the place to describe the existing work in more detail, pointing out the parts that you extend or improve and why you extend or improve these parts.

Design of Bidirectional Activation-based Learning algorithm (BAL) by \citet{farkas2013bal} is motivated by the biological plausibility of GeneRec. BAL inherits the learning rule of GeneRec \ref{eq:models-generec-learning-rule} and also the two phases. But unlike GeneRec, BAL aims to learn bidirectional mapping between inputs and outputs and for this purpose it uses four weights $W^{IH}$, $W^{HO}$, $W^{OH}$ and $W^{HI}$. The design of BAL is symmetric as shown in table~\ref{tab:bal-activation} and thus we avoid calling inputs, outpus, minus phase or plus phase. We rather choose \emph{forward} and \emph{backward} which could be interchanged. Note that the forward activations are denoted as $a^{\rm F}$ and backward activations as $a^{\rm B}$. 

\begin{table}
  \label{tab:models-bal-activation}
  \centering
  \begin{tabular}{|cccl|}
    \hline
    Layer & Phase & Net Input & Activation\\
    \hline
    \Bx & F & - & $x^{\rm F}_i$ = stimulus\\ [1ex]
    \Bh & F & \hspace{0.3cm}$\eta^{\rm F}_j = \sum_i w_{ij}^{IH}x^{F}_i$\hspace{0.3cm} & $h^{\rm F}_j = \sigma(\eta^{\rm F}_j)$\hspace{0.3cm}\\ [1ex]
    \By & F & $\eta^{\rm F}_k = \sum_j w_{jk}^{HO}h^{F}_j$ & $y^{\rm F}_k = \sigma(\eta^{\rm F}_k)$\\ [1ex]
    \hline
    \By & B & - & $y^{\rm B}_k$ = stimulus\\ [1ex]
    \Bh & B & $\eta^{\rm B}_j = \sum_k w_{kj}^{OH}y^{\rm B}_k$ & $h^{\rm B}_j = \sigma(\eta^{\rm B}_j)$\\ [1ex]
    \Bx & B  & $\eta^{\rm B}_i = \sum_j w_{ji}^{HI}h^{\rm B}_j$ & $x^{\rm B}_i = \sigma(\eta^{\rm B}_i)$\\
    \hline
  \end{tabular}
  \caption{Activation phases and states in BAL \citep{farkas2013bal}. Where \Bx is the first activation layer, i.e. \emph{front layer}, \By is the third activation layer, i.e. \emph{back layer}, $F$ means \emph{forward pass} and $B$ means \emph{backward pass}. Layers \Bx and \By are \emph{visible} and layer \By is hidden. Note that all non--stimulus units have learnable biases and their weights are updated in a same way as regular weights. } 
\end{table}

In the first phase called \emph{forward pass} the \emph{forward stimulus} is clamped and forward activations are computed. In the same way, in the second phase called \emph{backward pass} the \emph{backward stimulus} is clamped and backward activations are computed. We can imagine the backward pass as a reconstruction of the target pattern for the forward pass. For the learning rule the \emph{difference} between the forward pass and the backward pass is used: 
\begin{equation}
  \label{eq:models-bal-learning-rule-forward}
  \Delta w_{ij}^{\rm F} = \lambda \ a_i^{\rm F}(a_j^{\rm B} - a_j^{\rm F}),
\end{equation}
and for completeness we also provide the backward learning rule which is same as the forward learning rule~\ref{eq:models-bal-learning-rule-forward}: 
\begin{equation}
  \label{eq:models-bal-learning-rule-backward}
  \Delta w_{ij}^{\rm B} = \lambda \ a_i^{\rm B}(a_j^{\rm F} - a_j^{\rm B}). 
\end{equation}
Both forward~\ref{eq:models-bal-learning-rule-forward} and backward~\ref{eq:models-bal-learning-rule-backward} learning rules are same as the basic GeneRec learning rule~\ref{eq:models-generec-learning-rule}. We experimented with different learning learning rules \ref{sec:our-learning-rules}. 

 


 


% ==================== 13. Design and Implementation ========================
%You can divide this chapter in two sections: Design and Implementation.
%
% 1) Design – in design section you should describe your approach to solve the problem. The high level design of your solution and the modules, data structures and algorithms that you use.
%
% 2) Implementation – in implementation section you should mention the tools that you use to implement, the target environment (e.g. linux, windows). Limitations (e.g. buffer sizes, connections number).
%
%If necessary you can add additional sections such as Discussion to discuss or emphasize on the interesting design points.

%\newpage
%    %Design – in design section you should describe your approach to solve the problem. The high level design of your solution and the modules, data structures and algorithms that you use.

\section{Design} 

\subsection{Considered modifications}
Viacero z prezentovaných modifikácií boli aplikované na iné typy neurónových sietí. My sme sa zaoberali aplikáciami na Generec, analyzovali sme ich konvergenciu a aplikovali ich na štandardnej sade úloh. 

\subsubsection{Regression} 

\textit{Regression} means that, after one pass around the loop, instead of setting the activity of a visible unit, $i$, to be equal to its current total input, $x_i(2)$, as determined by 
$$x_j = \sum_i y_iw_{ji} - \theta_j,$$
we set its activity to be 
$$y_i(2) = \lambda y_i(0) + (1-\lambda)x_i(2)$$
where the regression, $\lambda$, is close to 1. Using high regression ensures that the visible units only change state slightly so that when the new visible vector is sent around the loop again on the second pass, it has very similar effects to the first pass \cite{hinton1988learning}.

\subsubsection{Ballard}
Merging several closed loops of recirculation. 

Since the same learning rule is used for both visivle and hidden units, there is no problem in applying it to systems in which some units are the visible units of one module and the hidden units of another \cite{hinton1988learning}. 

We do not have a formal analysis \cite{hinton1988learning}.
%D.H. Ballard Proc. American Association for Artificial Intelligence, Seatle, WA, 1987

\includegraphics[width=8cm]{img/ballard.png}

\subsubsection{Dropout}
When a large feedforward neural network is trained on a small training set,
it typically performs poorly on held-out test data. This \emph{overfitting} is greatly
reduced by randomly omitting half of the feature detectors on each training
case. This prevents complex co-adaptations in which a feature detector is only
helpful in the context of several other specific feature detectors. Instead, each
neuron learns to detect a feature that is generally helpful for producing the
correct answer given the combinatorially large variety of internal contexts in
which it must operate. Random \emph{dropout} gives big improvements on many
benchmark tasks and sets new records for speech and object recognition.
 \cite{hinton2012improving} (Abstract copied). 
 
 TODO: Read the article. 
 
Conceptually the idea is simple. In each activation phase turn off randomly a half of all hidden neurons. This should restrict coadaptation of neurons. It a desired quality as we want from each unit to represent a different feature. If the units can coadapt then it might occur that more units represent the same feature. 

\subsubsection{Deep architectures} 
TODO  \cite{bengio2009learning}.

 % I think it's more appropriate to use only "Methodology" in my case 
%\newpage
%    %Implementation – in implementation section you should mention the tools that you use to implement, the target environment (e.g. linux, windows). Limitations (e.g. buffer sizes, connections number).

\section{Implementation} 

   
% ==================== 14. Experimental Results and Analysis ========================
\newpage
    %Experimental Methodology – in this section you should describe the environment where you did the experiments, tools you used (compilers, libraries, profilers, simulators) and benchmarks that you used. Here you should tell what is your evaluation criteria (e.g. speedup) and metrics (e.g. throughput). Anything important that was made to conduct the experiments should be here (e.g. preparing traces for reproducing deterministic executions). If your experimental methodology has limitations you should mention them here (e.g. when using simulator you used small data input sets).

\section{Methodology and Design}

\subsection{Models}

\subsubsection{Previous} 

\begin{itemize} 
\item CHL
\item GeneRec 
\item BAL
\end{itemize}  

\subsubsection{Candidate selection} 
Hidden distance (over 70\%) over in triangle (68.3 \%). 

\subsubsection{Recirculation Bidirectional Activation-based Learning algorithm} 
TODO!: Analysis of fluctuation. 
TODO!: Try different lambda on layers. 
About 33\% success (lambda 0.3). 

\paragraph{Overview} 
(5th of March) 
How to generalize GeneRec to both ways? 
Experimenting with:
\begin{itemize} 
\item  a) classic generec - use only 3 matrices (which should work best for symmetric) 
\item  b) bothwards = f(hidden\_net\_from\_input + hidden\_net\_from\_output) 
\item  c) using GeneRec distinctly for forward and backward and combine them 
\end{itemize} 

\paragraph{Recirculation step} 
If no stationary point is found after MAX\_ITERATION then we set the result to avarage of the last two activations. This lead to 80\% less fluctuation. But still occured: 
\begin{itemize}
\item Big fluctuation: 0.9591299483462391
\item Not enough iterations: 20
\item Even if MAX\_ITERATION = 200 then max fluctuation still could be arbitrary and oscilating.
\end{itemize} 

\paragraph{Interesting} 
\begin{itemize} 
  \item oscilation in iteration could occur randomly (just in some epochs and it will completely change the network) 
  \item when using averages, it's less probable that a fluctuation will occur 
  \item setting MAX\_iteration much higher doesn't affect performance. About 50 is enough but 20 is not. 
\end{itemize} 
  
\paragraph{Conclusion} 
Symmetric weights haven't helped BAL (35\%) - no param selection
Using bothward GeneRec (60\%) - no param selection 
  Forgot to symmetric init -> no perceivable change 
In both cases (almost) no fluctuation 

Non-symmetric case, fluctuation in about 1/5 cases 
  About 30\% success 
  About 3-33 iterations needed to settle, very network dependent 
  About 50 - 5000 epochs needed 

IDEA: maybe bad implementation of backward
      using iterative activation has almost no reason (bothward is the generec idea) 

  
\subsubsection{Long training} 
even after 800,000 epochs there are some networks for which the error change

\subsubsection{Other} 
\begin{itemize} 
\item Momentum
\item Weight initialization 
\item Dynamic weight lambda (no change after few tries TODO test again) 
\item Batch mode (i.e. no shuffle and therefore deterministic)
\item Multilayer GeneRec - 42\% on handwritten digit recognition 
\item Dropout TODO (5\%,10\%,20\%,50\% chances)
\item TODO Noise 
\item TODO !Presenting in-out on which it has errors (based on rerun shuffle data). 
\end{itemize} 

\subsection{Evaluation methods} 

\subsubsection{Success rate} 

\begin{itemize}
\item Overall success. 
\item Bit success. 
\item Pattern success. 
\end{itemize} 

\subsubsection{Epochs needed for convergence} 



\subsection{Experiments}  

\subsubsection{4-2-4 Encoder} 

TODO: Write it as "Because BAL had significantly worse performance on this experiment we tried to analyse what are the reasons and propose improvements". 

To compare the performance of BAL with GeneRec, we ran tests using the well-
known 4-2-4 encoder task, following O’Reilly \cite{o1996bio}. We investigated the convergence
of BAL and the number of required training epochs as a function of the learning
rate. Fig. 1 shows the convergence success for 100 networks and the average
numbers of epochs needed. The simulations showed that convergence of BAL
depends on the learning rate, with the highest number of 65\% successful runs
achieved for $\lambda = 0.9$ \cite{farkas2013bal}. For comparison, O’Reilly \cite{o1996bio} reports 90\% success for basic GeneRec algorithm and 56\% for a symmetric modification of GeneRec and its
modification equivalent to CHL. In sum, probability of BAL convergence is lower
than that of basic GeneRec rule, but comparable to its symmetric versions. We
expect that the smaller number of successful runs is in both cases influenced by
the bidirectional nature of the weight update.

\subsubsection{Complex Binary Vector Associations} 

TODO: Write it as "Small scale graphical task to see the reconstruction" 
We evaluated the network performance on n–to–1 data associations, motivated
by the sensory-motor mappings between distributed patterns. For this purpose
we created low-dimensional sparse binary codes, 16-dimensional vectors ($4 \times 4$
map) with k = 3 active units with n = 4. For each target (y), these four
patterns (x) were assumed to have nonzero overlap. Again, we searched for
optimal $\lambda$ and $n_H$ (Fig. 5). The best performance was achieved using $\lambda$ $\approx$ 1. We
can observe that the ambiguity in the data association causes the network to
produce errors in B direction. For the best $\lambda$ the networks yielded patSuccB $\approx$
4\% and $bitSucc^B \approx 86\%$, which means that the networks made small errors in
most patterns. This could be expected since the network cannot know which of
the four (x) patterns is to be reconstructed. It is known, that a network trained
to associate more binary target patterns with one pattern tends to produce a
mesh of outputs, weighed by their frequency of occurrence in the training set.
Examples of network outputs are illustrated in Fig. 6.

\begin{center} 
\includegraphics{img/cbva_back_repre.png} 
\end{center} 

\subsubsection{Hand-written Digits} 
TODO: Write overview as "High dimensional graphical task to see backward representation and compare it with a bunch of other models" 
TODO: Compare results: \url{http://yann.lecun.com/exdb/mnist/} 
Data from MNIST database \cite{lecun1998gradient}. 




\newpage
    %Implementation – in implementation section you should mention the tools that you use to implement, the target environment (e.g. linux, windows). Limitations (e.g. buffer sizes, connections number).

\section{Implementation} 

\newpage
    %Experimental Results and Analysis – in this section you should show the quantitative results – charts and tables. Analyze the results by explaining and highlighting what is important on them in terms of your goals and what is bad. You should explain the strange results too.

\section{Results and Analysis} 

TODO: success / learning rate 
TODO: epochs / learning rate 
TODO: success / epochs 
TODO: bitSucc and patSucc
TODO: success / hidden layer size 

\subsection{TODO} 

\subsubsection{Measure - Weight decay}
A commonly-used bias or regularizing function is weight decay (e.g., Hinton, 1989a; Weigend et al., 1991).
We implemented two commonly-used forms of weight decay in the Bp and GeneRec networks, simple
weight decay and weight-elimination weight decay (Weigend et al., 1991). In simple weight decay, a con-
stant fraction of the weight value is subtracted at each weight update, and weight-elimination is similar
except that the rate of decay is a more complex function of the weight such that larger weights suffer rela-
tively less decay than smaller ones (supporting a prior assumption of a bimodal distribution of weight values
— one population of larger weights that are actually useful and another of near-zero weights that are not
useful; see Weigend et al., 1991 for details).
The results with these forms of weight decay for the 100 hidden unit network are shown in Figure 5.
Although a small amount (.002; smaller amounts had progressively smaller effects) of simple weight de-
cay appears to reliably improve generalization performance in both Bp and GeneRec, the difference is not
substantial. The weight-elimination version of weight decay always appears to impair, rather than improve,
performance. Although the specific forms of weight decay explored here were not overly successful in
this task, it is possible that other forms might perform better. Nevertheless, most forms of weight decay
\cite{o2001generalization} 

\subsubsection{Dynamic learning rate} 
%TODO citation 

\subsubsection{Measure - Weight patterns} 
An examination of the weights in the trained networks clearly shows why generalization is impaired in the
interactive network (Figure 6). The units have not carved the input/output mapping into separable subsets
that can be independently combined for the novel testing items — instead, each unit participates in the
input/output mapping for multiple slots. Although this is true for both the backpropagation and GeneRec
networks, it is particularly damaging for the interactive GeneRec network because of its attractor dynamics.
In contrast, the feedforward backpropagation network is still capable of producing a roughly linear combi-
nation of hidden unit activations that yields reasonable (though far from perfect) levels of generalization. \cite{o2001generalization} 

Figure 7: Average pairwise overlap (normalized dot product or cosine) between hidden patterns corresponding to
inputs that differ by a) 75\% (1 out of 4 slots different) and b) 50\% (2 out of 4 slots different). Feedforward back-
propagation (Bp) remains much closer to a linear response level (75\% and 50\% hidden similarity, respectively) after
training compared to interactive GeneRec, which shows evidence of attractor dynamics pulling the network away from
a linear, combinatorial response to the inputs.


    
% ==================== 16. Conclusion ========================
% This chapter should conclude on your contribution. It should highlight the key results from the research work. In this section you should avoid mentioning new terms and statements not discussed throughout the main text. Also general aspects of the research work shouldn’t be repeated here. Conclusion shouldn’t be the abstract written in past tense. The conclusion should derive the important facts out of your work and results that you obtained.
\newpage
    \section*{Conclusion}
\markboth{CONCLUSION}{}
\addcontentsline{toc}{section}{Conclusion}
\label{sec:conclusion} 

Still, many work should be done. 


==ZAVER 
\begin{itemize} 
\item   spravili sme toto toto
\item   toto je otvorene
\item   toto vyzera tazko
\item   tymto by sme sa zaoberali dalej 
\begin{itemize}
\item ako rozlisit uspesne a neuspesne od inicializacie (v batch mode), t.j. binarny klasifikator na good/bad vah 
\end{itemize} 

\end{itemize} 

%What left unfinished from your work? What are you future plans to develop better your work?

\subsection*{Future Work} 
Weight initialization: train on the dataset (initial weights + success). 

Analyse convergence 
(TODO ref hidden activations) \ref{sec:convergence} 

Weight decay: 
\ref{sec:weight-decay} 

Recirc close weight: 
\begin{equation}
o_k = \alpha t_k + (1-\alpha)f(\eta_k). 
\end{equation} 

Try TLR also with GeneRec (and BP?).  \\ 

\label{sec:future-dlr} 
Dynamic learning rate \ref{sec:our-dynamic-lambda}. 

Tricks with BP \citep{lecun2012efficient}. 




    
% ==================== 17. Future Work ========================
%What left unfinished from your work? What are you future plans to develop better your work?
\newpage
    %What left unfinished from your work? What are you future plans to develop better your work?

\section*{Future Work} 


% ==================== 18. Bibliography ========================
%Put a list of references.
\newpage
    %Put a list of references.

%TODO check author names (same) 
%TODO check journal names (always big) 
%TODO add publisher (using note?) 

\renewcommand{\refname}{Bibliography}
\phantomsection
\addcontentsline{toc}{section}{Bibliography}

\bibliographystyle{apalike} 
\bibliography{main} 


    
% ==================== 19. Assignment ========================
%At the end of your thesis you can attach resources such as source code (or something like ASCII code table) that would improve the completeness of your thesis.
\newpage
    %At the end of your thesis you can attach resources such as source code (or something like ASCII code table) that would improve the completeness of your thesis.

\section*{Appendix A}
\appendix
\addcontentsline{toc}{section}{Appendix A}
\markboth{Appendix A}{}

%TODO crucial parts of the implementation 
%TODO additional tables, measures 


\end{document}
